\documentclass{amsart}

\usepackage[utf8]{inputenc}
\usepackage{amsmath, amssymb, amsthm, mathtools, textcomp, gensymb}
%\usepackage[margin=1in]{geometry}
\usepackage{cancel}
\usepackage{changepage}
%\usepackage{fancyhdr}
\usepackage{caption}
\usepackage{graphicx, hang, calc}
\usepackage{xcolor, wrapfig}
\usepackage{tkz-euclide}
\usepackage{tikz-3dplot, pgfplots}
\usepackage{float}
\usepackage{centernot}

\pgfplotsset{compat=1.18}

\graphicspath{{graphics/}}



%\pagestyle{fancy}
%\lhead{Arcedera, Marc Russel C. \\ Mathematics 140}
%\rhead{1\textsuperscript{st} Semester A.Y. 2021-2022 \\ Reviewer}

\renewcommand{\baselinestretch}{1}
\setlength{\parindent}{0mm}
\setlength{\headheight}{25pt}
\DeclarePairedDelimiter{\norm}{\lVert}{\rVert}
\DeclarePairedDelimiter{\vect}{\langle}{\rangle}

\title{Exercises in Analysis I (3rd Ed), Tao}
\author{Arcedera, Marc Russel C.}

\newtheorem{theorem}{Theorem}[section]
\newtheorem{corollary}{Corollary}[theorem]
\newtheorem{lemma}[theorem]{Lemma}
\newtheorem{proposition}[theorem]{Proposition}

\newtheorem{innercustomthm}{Theorem}
\newenvironment{customthm}[1]
  {\renewcommand\theinnercustomthm{#1}\innercustomthm}
  {\endinnercustomthm}
  
\newenvironment{myindent}
{\par\leftskip1cm\relax\rightskip1cm\relax\itshape}
{\par\leftskip0cm\relax\rightskip0cm\relax}

%\theoremstyle{remark}

\theoremstyle{definition}
\newtheorem{definition}{Definition}[section]
\newtheorem{remark}{Remark}[section]
\newtheorem*{Remarks}{Remarks}
\newtheorem*{example}{Example}
\newtheorem*{Examples}{Examples}


\usepackage{blindtext}


\newcommand{\C}{\mathbb{C}}
\newcommand{\R}{\mathbb{R}}
\newcommand{\Q}{\mathbb{Q}}
\newcommand{\Z}{\mathbb{Z}}
\newcommand{\N}{\mathbb{N}}
\newcommand{\F}{\mathbb{F}}

\newcommand{\soln}{\newline\textit{Solution.} }

%for "formal addition" in the textbook
\newcommand{\pls}{\mathop{\hspace{-2pt}+\hspace{-3.5pt}+{}}}
\newcommand{\mns}{\hspace{-2pt}-\hspace{-4.1pt}-{}}
\newcommand{\dvd}{//}
\newcommand{\LIM}{\mathrm{LIM}_{n\to\infty}\,}

\newcommand{\set}[1]{\left\{#1\right\}}


\begin{document}
\maketitle
\tableofcontents
\section{Introduction}
\textit{There are no exercises in this Chapter.}


\section{Starting at the beginning: the natural numbers}

\textbf{The Peano Axioms} \addtocounter{subsection}{1}
\textit{There are no exercises in this section.} \\

\textbf{Addition} \addtocounter{subsection}{1}

\subsubsection{} Prove Proposition 2.2.5. (Hint: fix two of the variables and induct on the third.) \\
\soln We will use induction on $c$, keeping $a$ and $b$ fixed. For the base case, we have $(a+b)+0=a+(b+0)$. By lemma 2.2.2, $(a+b)+0=a+b$. For the right hand side, we have $b+0=b$ by lemma 2.2.2 again so $a+(b+0)=a+b$ which settles the base case. Suppose inductively that $(a+b)+c=a+(b+c)$. We have to show that $(a+b)+c\pls=a+(b+c\pls)$. For the left side, using lemma 2.2.3, $(a+b)+c\pls=((a+b)+c)\pls$. For the right side, using lemma 2.2.3 again, $a+(b+c\pls)=a+(b+c)\pls=(a+(b+c))\pls$. By our inductive hypothesis, we must have $((a+b)+c)\pls=(a+(b+c))\pls$ which closes the induction. \\

\subsubsection{} Prove Lemma 2.2.10. (Hint: use induction.)\\
\soln We induct on $a$. Note that $0$ is not the successor of any natural number so we start the base case on $a=1$. We have $0\pls=1$ by definition. Suppose that $n$ is a natural number such that $n\pls=1$, then by axiom 2.4, $n=0$. Suppose inductively that there exists a unique natural number $b$ such that $b\pls=a$. We are to show that there exists a unique natural number $c$ such that $c\pls=a\pls$. By axiom 2.4, we have $c=a$ and by our inductive hypothesis, $c=b\pls$. The uniqueness of $b$ guarantees the uniqueness of $c$---since by axiom 2.4, for any other natural number $m\neq b$, $m\pls\neq b\pls =c$. This closes the induction. \\

\subsubsection{} Prove Proposition 2.2.12. (Hint: you will need many of the preceding propositions, corollaries, and lemmas.) \\
\soln 
\begin{enumerate}
\item[(a)] Observe that $0$ is a natural number by axiom 2.1 and $a=a+0$ by lemma 2.2.2, so $a\geq a$. 
\item[(b)] Suppose $a\geq b$ and $b\geq c$. By definition, there exists natural numbers $n,m$ such that $a=b+n$ and $b=c+m$. It follows that $a=(c+m)+n=c+(m+n)$ since addition is associative. It should be clear that $m+n$ is a natural number---if both $m,n=0$ then $m+n=0$ which is a natural number by axiom 2.1 and otherwise, proposition 2.2.8 asserts that $m+n$ is positive and hence a natural number. Thus, $a\geq c$.
\item[(c)] Suppose that $a\geq b$ and $b\geq a$. By definition, there exists natural numbers $n,m$ such that $a=b+n$ and $b=a+m$. It follows that $a=a+(m+n)$ and by proposition 2.2.6 (cancellation law for addition), $m+n=0$. Then by corollary 2.2.9, $m=0$ and $n=0$. Finally, by lemma 2.2.2, $b+0=b$ and $a+0=a$ so $a=b$.
\item[(d)] $(\implies)$ Let $a\geq b$. We induct on $c$. For the base case, $c=0$, $a=a+0$ and $b=b+0$ and by assumption, $a\geq b$. Suppose inductively that $a+c\geq b+c$, then there exists a natural number $n$ such that $a+c=b+c+n$ (we can omit the parentheses since addition is associative). We are to show that $a+c\pls\geq b+c\pls$. We have $a+c\pls=(a+c)\pls=(b+c+n)\pls$ by lemma 2.2.3. Using the commutativity of addition (prop 2.2.4) and lemma 2.2.3 again, $(b+c+n)\pls=(b+n+c)\pls=b+n+c\pls=(b+c\pls)+n$. Thus, $a+c\pls=(b+c\pls)+n$ which is the definition of $a+c\pls\geq b+c\pls$. This closes the induction. $(\impliedby)$ Conversely suppose that $a+c\geq b+c$, then there exists a natural number $n$ such that $a+c=b+c+n$. Using the cancellation law for addition, we have $a=b+n$ and hence $a\geq b$. 
\item[(e)] 
\end{enumerate}

\subsubsection{} Justify the three statements marked (why?) in the proof of Proposition 2.2.13. \\
\soln 
\begin{enumerate}
    \item $0\leq b$ for all $b$: We induct on $b$. Clearly, $0\leq 0$ since $0=0+0$ (lmao) which proves the base case. Suppose inductively that $0\leq b$. We have to show that $0\leq b\pls$. By our induction hypothesis, we have $b=0+n$ for some natural number $n$. By axiom 2.4 and lemma 2.2.2, we have $b\pls=(0+n)\pls=0+n\pls$. Note that $n\pls$ is also a natural number by axiom 2.2. Hence, $0\leq b\pls$ and this closes the induction.
    \item If $a>b$, then $a\pls>b$: Suppose $a>b$, then $a=b+n$ for some natural number $n$ and $a\neq b$. By axiom 2.4 and lemma 2.2.2, we have $a\pls=(b+n)\pls=b+n\pls$. Axiom 2.2 asserts that $n\pls$ is also a natural number. Thus, $a\pls\geq b$. Now, suppose on the contrary that $a\pls=b$. Then, $b+n\pls=b$ and by the cancellation law for addition, $n\pls=0$ which is a contradiction to axiom 2.3. Hence, $a\pls>b$.
    \item If $a=b$, then $a\pls>b$: Suppose $a=b$, then $a\pls=b\pls$ by axiom 2.3. Note that $b\pls=b\pls+0=(b+0)\pls=b+0\pls=b+1$. Thus, $a\pls=b+1$ so $a\pls\geq b$. Moreover, we cannot have $a\pls=b$ because otherwise, $a\pls=a$ which contradicts axiom 2.5 (PMI) as such: By axiom 2.2, $0$ is not the successor of any natural number so $0\pls\neq 0$. Suppose inductively that $a\pls\neq a$. We need to show that $(a\pls)\pls\neq a\pls$. Suppose otherwise, then by axiom 2.4, $a\pls=a$, a contradiction. Hence, $(a\pls)\pls\neq a\pls$ which closes the induction. Thus, $a\pls>b$.
\end{enumerate}

\subsubsection{} Prove Proposition 2.2.14. (Hint: define $Q(n)$ to be the property that $P(m)$ is true for all $m_0\leq m<n$; note that $Q(n)$ is vacuously true when $n<m_0$.) \\
\soln Let $Q(n)$ be the property that $P(m)$ is true for all $m_0\leq m<n$. We induct on $n$. For the base case, we have $Q(m_0\pls)$ which is true because $m_0\leq m<m_0\pls$ implies that $m=m_0$ only and $P(m_0)$ is vacuously true. Suppose inductively that $Q(n)$ is true for some natural number $n$. We are to show that $Q(n\pls)$ is also true. Since $Q(n)$ is true, we have $P(m)$ is true for all $m_0\leq m<n$. By assumption, it follows that $P(n)$ is also true. Thus, we have shown that $P(m)$ is true for all natural numbers $m_0\leq m \leq n <n\pls$. Note that there is no natural number between $n$ and $n\pls$ so $P(m)$ is true for all natural numbers $m_0\leq m<n\pls$. Hence, $Q(n\pls)$ is true which closes the induction.\\

\subsubsection{} Let $n$ be a natural number, and let $P(m)$ be a property pertaining to the natural numbers such that whenever $P(m\pls)$ is true, then $P(m)$ is true. Suppose that $P(n)$ is also true. Prove that $P(m)$ is true for all natural numbers $m\leq n$; this is known as the \textit{principle of backwards induction}. (Hint: apply induction to the variable $n$.) \\
\soln We induct on $n$. For the base case, $n=0$, it should be clear that $m\leq 0$ implies that $m=0$ since $m$ is a natural number so $P(n)=P(m)$ holds. Suppose inductively that $P(n)$ is true for some natural number $n$. We have to show that $P(n\pls)$ also holds. We know that $P(m)$ is true for all natural numbers $m\leq n$ \textit{not finished}


\subsection{Multiplication}

\subsubsection{} Prove Lemma 2.3.2. (Hint: modify the proofs of Lemmas 2.2.2, 2.2.3 and Proposition 2.2.4.) \\
\soln (a) We first show that $n\times 0=0$ for any natural number $n$ using induction. The base case $0\times 0=0$ follows by definition of multiplication since $0$ is a natural number. Suppose inductively that $n\times0=0$. We are to show that $(n\pls)\times0=0$ but by definition of multiplication, $(n\pls)\times0=(n\times0)+0=0+0=0$. This closes the induction. (b) Next, we need to show that for any natural numbers $n$ and $m$, we have $n\times(m\pls)=(n\times m)+n$. We induct on $n$. The base case easily follows by definition, $0\times (m\pls)=(0\times m)+0$. Suppose inductively that $n\times (m\pls)=(n\times m)+n$. We have 
\begin{align*}
(n\pls)\times (m\pls) = (n\times (m\pls))+m\pls &=(n\times m)+n+m\pls \\
&=(n\times m)+m+n\pls \\
&=((n\pls)\times m)+n\pls
\end{align*}
which is what we want. This closes the induction. (c) We finally prove the claim that multiplication is commutative. We will use induction on $n$. The base case, $n=0$, follows by definition $0\times m=0$ and by (a) $m\times 0=0$. Suppose inductively that $n\times m=m\times n$. We are to show that $(n\pls)\times m=m\times (n\pls)$. We have $(n\pls)\times m=(n\times m)+m$ by definition. Using the induction hypothesis and $(b)$, we get $(n\times m)+m=(m\times n)+m=m\times (n\pls)$. This closes the induction. \\

\subsubsection{} Prove Lemma 2.3.3. (Hint: prove the second statement first.) \\
\soln 
\begin{enumerate}
\item[(a)] As hinted, we prove the second statement first: \textit{if $n$ and $m$ are both positive, then $nm$ is also positive}. We will induct on $n$ keeping $m$ fixed. By assumption, we have $n\neq0$ and $m\neq0$ so we start the base case at $n=1$. Observe that $1\times m=(0\pls)\times m=(0\times m)+m=0+m=m$ which is positive. Suppose inductively that $nm$ is positive, then we have $(n\pls)m=nm+m$ which is also positive. This closes the induction. \\ 
\item[(b)] Next, we prove the first statement: \textit{Let $n,m$ be natural numbers. Then $n\times m=0$ iff at least one of $n,m$ is equal to zero}. $(\implies)$ This is the contrapositive of (a). $(\impliedby)$ Suppose at least one of $n,m$ is equal to zero, then it easily follows by definition (and commutativity of multiplication) that $nm=0$. \\
\end{enumerate}

\subsubsection{} Prove Proposition 2.3.5. (Hint: modify the proof of Proposition 2.2.5 and use the distributive law.) \\
\soln 


\subsubsection{} Prove the identity $(a+b)^2=a^2+2ab+b^2$ for all natural numbers $a,b$. \\
\soln Let $a,b$ be natural numbers. we have
\begin{align*}
    (a+b)^2=(a+b)^{1\raisebox{0.8pt}{\scalebox{0.7}{\pls}}}=(a+b)^1(a+b)&=(a+b)(a+b)\\
    &=a(a+b)+b(a+b) \\
    &=aa+ab+ba+bb \\
    &=a^2+ab+ab+b^2 \\
    &=a^2+(1+1)ab+b^2 \\
    &=a^2+2ab+b^2
\end{align*}
as desired. \\

\subsubsection{} Prove Proposition 2.3.9. (Hint: fix $q$ and induct on $n$.)

\newpage

\section{Set Theory}

\subsection{Fundamentals} 

\subsubsection{} Show that the definition of equality in Definition 3.1.4 is reflexive, symmetric, and transitive. \\
\soln Reflexive: For any object $x$, $x\in A$ iff $x\in A$ so $A=A$. (The two statements are equivalent because, well, they are identical.) \\
Symmetric: Let $A,B$ be sets and suppose that $A=B$. By definition, $x\in A$ iff $x\in B$ which is logically equivalent to $x\in B$ iff $x\in A$. Thus, $B=A$. \\
Transitive: Let $A,B,C$ be sets and suppose that $A=B$ and $B=C$. By definition, $x\in A$ iff $x\in B$, and $x\in B$ iff $x\in C$ which is logically equivalent to saying $x\in A$ iff $x\in C$. Hence, $A=C$ as desired. \\

\subsubsection{} Using only Definition 3.1.4, Axiom 3.1, Axiom 3.2, and Axiom 3.3, prove that the sets $\emptyset$, $\set{\emptyset}$, $\set{\set{\emptyset}}$, and $\set{\emptyset, \set{\emptyset}}$ are all distinct (i.e., no two of them are equal to each other). \\
\soln We first note that Axiom 3.1 asserts that sets themselves are objects so a set can be an element of another set. Moreover, Axiom 3.2 asserts the existence of the empty set $\emptyset$. It follows that the empty set $\emptyset$ is also an object and by Axiom 3.3, there exists a set whose only element is $\emptyset$, namely $\set{\emptyset}$. Similarly, by Axiom 3.1 and 3.3, the sets $\set{\set{\emptyset}}$ and $\set{\emptyset,\set{\emptyset}}$ exist. Axiom 3.2 guarantees that $\emptyset$ is not equal to any of the other three sets because there exists an object in each of them. In particular, we note that $\emptyset\neq\set{\emptyset}$. Since $\set{\emptyset}\in\set{\set{\emptyset}}$ and $\set{\emptyset}\in\set{\emptyset,\set{\emptyset}}$ but $\set{\emptyset}\notin\set{\emptyset}$, we conclude that $\set{\emptyset}$ is not equal to $\set{\set{\emptyset}}$ or $\set{\emptyset,\set{\emptyset}}$. Finally, since $\emptyset\in\set{\emptyset,\set{\emptyset}}$ but $\emptyset\notin\set{\set{\emptyset}}$, we also conclude that $\set{\set{\emptyset}}\neq \set{\emptyset,\set{\emptyset}}$. Thus, the given sets are distinct from each other. \\

\subsubsection{} Prove the remaining claims in Lemma 3.1.13. \\
\soln (a) \textit{If $a$ and $b$ are objects, then $\set{a,b}=\set{a}\cup\set{b}$}: We need to show that every element of $\set{a,b}$ is also an element of $\set{a}\cup\set{b}$. We have $a\in\set{a}$ so by Axiom 3.4, $a\in\set{a}\cup\set{b}$. Similarly, since $b\in\set{b}$ we have $b\in\set{a}\cup\set{b}$. Clearly, $a,b\in\set{a,b}$. Thus, by Definition 3.1.4, $\set{a,b}=\set{a}\cup\set{b}$. \\

\textit{If $A,B,C$ are sets, then the union operation is commutative (i.e., $A\cup B=B\cup A$)}: Suppose $x\in A\cup B$. Then $x\in A$ or $x\in B$ which is logically equivalent to saying $x\in B$ or $x\in A$ so $x\in B\cup A$. Similarly, suppose that $y\in B\cup A$. Then $y\in B$ or $y\in A$ and thus $y\in A\cup B$. Hence, $A\cup B=B\cup A$ as desired. \\

\textit{$A\cup A=A\cup\emptyset=\emptyset\cup A=A$}: Note that the statement $x\in A$ or $x\in A$ is logically equivalent to simply saying $x\in A$. Thus, $x\in A$ implies that $x\in A\cup A$ and $x\in A\cup A$ implies that $x\in A$ so $A=A\cup A$. Next, observe that $x\in\emptyset$ is a contradiction (meaning it is never true) so $x\in A$ is logically equivalent to $x\in A$ or $x\in\emptyset$. So $x\in A$ implies that $x\in A\cup\emptyset$ and $x\in A\cup\emptyset$ implies that $x\in A$. Thus, $A=A\cup\emptyset$ and by commutativity of $\cup$ and transitivity of set equality, we have $A\cup A=A\cup\emptyset=\emptyset\cup A=A$. \\

\subsubsection{} Prove the remaining claims in Proposition 3.1.18. \\
\soln 
\begin{enumerate}
\item[(a)] \textit{If $A\subseteq B$ and $B\subseteq A$, then $A=B$}: Suppose $A\subseteq B$ and $B\subseteq A$. Let $x$ be an arbitrary element of $A$. Since $A\subseteq B$, we must have $x\in B$. Similarly, let $y$ be an arbitrary element of $B$. Since $B\subseteq A$, we must also have $y\in A$. By Definition 3.1.4, $A=B$. \\
\item[(b)] \textit{If $A\subsetneq B$ and $B\subsetneq C$ then $A\subsetneq C$}: Suppose $A\subsetneq B$ and $B\subsetneq C$. We need to show that $A\subseteq C$ and $A\neq C$. The first claim has already been proven so we only need to show that the two sets are not equal. Since $A\subsetneq B$, $B\subsetneq C$, we have have $A\neq B$ and $B\neq C$, respectively. By transitivity of set equality, it follows that $A\neq C$, as desired. \\
\end{enumerate}

\subsubsection{} Let $A,B$ be sets. Show that the three statements $A\subset B$, $A\cup B=B$, and $A\cap B=A$ are logically equivalent (any one of them implies the other two). \\
\soln 
\begin{enumerate}
\item[(a)] $A\subset B\implies A\cup B=B$: Suppose $A\subset B$, then $x\in A$ implies that $x\in B$. Let $x\in A\cup B$ which means that $x\in A$ or $x\in B$. Since $A\subset B$, it follows that $x\in B$ or $x\in B$ which is equivalent to saying $x\in B$. Thus, $x\in B$. Let $y\in B$. It follows that $y\in A$ or $y\in B$ (this is true because we have assumed $y\in B$ to be true) so $y\in A\cup B$. Hence, $A\cup B=B$. \\
\item[] $A\subset B\impliedby A\cup B=B$: Suppose $A\cup B=B$ and suppose on the contrary that $A\not\subseteq B$, i.e., there exists $x\in A$ such that $x\notin B$ holds. This immediately contradicts our assumption since there exists an element $x$ in $A$, and hence in $A\cup B$, but not in $B$, meaning that $A\cup B\neq B$. Thus, proved. \\
\item[(b)] $A\subset B\implies A\cap B=A$: Suppose $A\subset B$. Let $x$ be an arbitrary element of $A\cap B$, then $x\in A$ by definition of intersection of sets. Let $y$ be an arbitrary element of $A$. Since $A\subset B$, we have $y\in A$ implies $y\in B$. Note that $y\in A$ is equivalent to $y\in A$ and $y\in A$ which implies $y\in A$ and $y\in B$. Thus, $y\in A\cap B$ and hence $A\cap B=A$. \\
\item[] $A\subset B\impliedby A\cap B=A$: Suppose $A\cap B=A$ and suppose on the contrary that $A\not\subset B$, i.e., there exists $x\in A$ such that $x\notin B$. This immediately contradicts our assumption because $x\notin B$ implies that $x\notin A\cap B$ which follows that $A\cap B\neq A$. Thus, proved. \\
\item [] Since logical equivalence is transitive, we have \[A\subset B\iff A\cup B=B\iff A\cap B=A\] as desired. \\
\end{enumerate} 

\subsubsection{} Prove Proposition 3.1.28. (Hint: one can use some of these claims to prove others. Some of the claims have also appeared previously in Lemma 3.1.13.) \\
\soln
\begin{enumerate}
\item[(a)] We know that $A\cup\emptyset=A$ by Lemma 3.1.13 (\textit{Exercise} 3.1.3). Let $x$ be an arbitrary element of $A\cap\emptyset$ so $x\in A$ and $x\in\emptyset$. However, $x\in\emptyset$ can never be true for any object $x$ by Axiom 3.2 so the statement $x\in A$ and $x\in\emptyset$ will never be true as well. Thus, $x\notin A\cap\emptyset$ for any object $x$. Hence, by Axiom 3.2, $A\cap\emptyset$ is the empty set---or symbolically, $A\cap\emptyset=\emptyset$. \\
\item[(b)] Since $A\subseteq X$, if we suppose that $A\neq X$ then we have $A\cup X=X$ and $A\cap X=A$ by \textit{Exercise} 3.1.5. Otherwise, if $A=X$, the claim is trivially true---see item (c) below. \\
\item[(c)] Let $x$ be an arbitrary element of $A$. Logically, $x\in A$ is equivalent to $x\in A$ \underline{and} $x\in A$, as well as $x\in A$ \underline{or} $x\in A$ (the use of \textit{and} and \textit{or} here are as logical connectives, underlined for emphasis). Hence $A=A\cap A=A\cup A$. \\
\item[(d)] We have already shown that $A\cup B=B\cup A$ in Lemma 3.1.13 (\textit{Exercise} 3.1.3). Let $x\in A\cap B$, then $x\in A$ and $x\in B$. This is equivalent to saying $x\in B$ and $x\in A$ so $x\in B\cap A$. Similarly, let $y\in B\cap A$, then $y\in B$ and $y\in A$ iff $y\in A$ and $y\in B$. Thus, $y\in A\cap B$ and we have $A\cap B=B\cap A$. \\
\item[(e)] Associativity of the union operation was already shown in Lemma 3.1.13. Let $x$ be an arbitrary element of $(A\cap B)\cap C$. Then, by Definition 3.1.23, $x\in A\cap B$ and $x\in C$ and again by Definition 3.1.23, $x\in A$ and $x\in B$ and $x\in C$. By the two latter conjunctions, we have $x\in B\cap C$ and since $x\in A$ as well, we have $x\in A\cap (B\cap C)$. The reverse direction can be shown using a similar argument. Hence, $(A\cap B)\cap C=A\cap (B\cap C)$. \\
\item[(f)] Let $x\in A\cap(B\cup C)$. We have $x\in A$ and $(x\in B$ or $x\in C)$. We split this into two cases. (1) If $x\in B$, then we have $x\in A$ and $x\in B$ which means that $x\in A\cap B$ and so $x\in (A\cap B)\cup(A\cap C)$. (2) If $x\in C$, then we have $x\in A$ and $x\in C$ which means that $x\in A\cap C$ and so $x\in (A\cap B)\cup(A\cap C)$. In either case we have $x\in (A\cap B)\cup(A\cap C)$. For the reverse direction, let $y\in (A\cap B)\cup(A\cap C)$. Then $y\in (A\cap B)$ or $y\in(A\cap C)$ and thus, $(y\in A$ and $y\in B)$ or $(y\in A$ and $y\in C)$. We split this into two cases again. (1) For the former, since $y\in B$, it follows that $y\in B\cup C$ and because $x\in A$ as well, we have $y\in A\cap(B\cup C)$. (2) Similarly for the latter case, $y\in B\cup C$ and $y\in A$ so $y\in A\cap(B\cup C)$. Either case also hold and we are done. Thus, $A\cap(B\cup C)=(A\cap B)\cup(A\cap C)$. \\
\item[(g)] We first want to note that, logically, the statement $x\in A$ or $(x\in X$ and $x\notin A)$ is equivalent to $(x\in A$ or $x\in X)$ and $(x\in A$ or $x\notin A)$. Note that the latter disjunction is a tautology so the statement simplifies to $x\in A$ or $x\in X$. In summary, 
\[
x\in A \text{ or } (x\in X \text{ and }  x\notin A) \iff  x\in A \text{ or } x\in X
\]
for any object $x$. Now suppose $x\in A\cup(X\setminus A)$. Then by Axiom 2.4, $x\in A$ or $x\in X\setminus A$ and then by Definition 3.1.27 we have $x\in A$ or $(x\in X$ and $x\notin A)$. By what we have shown above, this is equivalent to $x\in A$ or $x\in X$. Lastly, since $A\subseteq X$, $x\in A$ implies $x\in X$. Thus, $x\in X$. For the reverse direction, suppose $y\in X$, then the statement $y\in X$ or $y\in A$ is also true. We know that this is equivalent to $y\in A$ or $(y\in X$ and $y\notin A)$ and so $y\in A\cup(X\setminus A)$. Hence, $A\cup(X\setminus A)=X$. \\

\noindent Suppose $x\in A\cap(X\setminus A)$. Using similar arguments, it can be shown that this is equivalent to saying $x\in A$ and $x\in X$ and $x\notin A$ (no need for parentheses since the logical \textit{and} operator is associative). However, $x\in A$ and $x\notin A$ is a contradiction---meaning it can never be true---so no such object $x$ exists. By Axiom 3.2, $A\cap(X\setminus A)=\emptyset$. \\
\item[(h)] 
\end{enumerate}

\textit{Remark.} Notice that these properties of set operations seem analogous to the properties of logical connectives. Indeed, this is because set operations were defined using logical connectives e.g., the union operator is analogous to logical disjunction (\textit{or}), the intersection operator is to logical conjunction (\textit{and}). \\

\subsubsection{} Let $A,B,C$ be sets. Show that $A\cap B\subseteq A$ and $A\cap B\subseteq B$. Furthermore, show that $C\subseteq A$ and $C\subseteq B$ if and only if $C\subseteq A\cap B$. In a similar spirit, show that $A\subseteq A\cup B$ and $B\subseteq A\cup B$, and furthermore that $A\subseteq C$ and $B\subseteq C$ if and only if $A\cup B\subseteq C$. 





\subsection{Russell's paradox (Optional)} 

\subsubsection{} Show that the universal specification axiom, Axiom 3.8, if assumed to be true, would imply Axioms 3.2, 3.3, 3.4, 3.5, and 3.6. (If we assume that all natural numbers are objects, we also obtain Axiom 3.7.) Thus, this axiom, if permitted, would simplify the foundations of set theory tremendously (and can be viewed as one basis for an intuitive model of set theory known as "naive set theory"). Unfortunately, as we have seen, Axiom 3.8 is "too good to be true"! \\
\soln Suppose Axiom 3.8 holds. 
\begin{enumerate}
    \item[(3.2)] Let $P(x)$ be the property pertaining to an object $x$ such that $P(x)$ is always false. (One example would be $P(x): x\neq x$ however we may not always be sure whether equality is defined for $x$.) Using Axiom 3.8, we can construct the set $\set{x: P(x)}$. $P(y)$ is true iff $y\in\set{x:P(x)}$ but since $P(y)$ is always false, $y\notin\set{x:P(x)}$ for any object $y$. This statement is equivalent to Axiom 3.2 which is what we want. We can name this set as the empty set denoted $\emptyset$. \\
    \item[(3.3)] Suppose $a$ is an object and let $P(x): x=a$ (or if equality is not defined for the object $a$, $P(x): x$ is the object $a$). Using Axiom 3.8, we construct the set $\set{x: P(x)}=\set{x:x=a}$. Note that $P(y)$ is true iff $y=a$ iff $y\in\set{x:x=a}$ so the only element of $\set{x:x=a}$ is $a$. Thus, we have constructed the singleton set $\set{a}$. Similarly, suppose $a$ and $b$ are objects. Let $Q(x): x=a \text{ or } x=b$ (or if equality is not defined for the objects $a$ and $b$, $Q(x): x$ is one of the objects $a$ or $b$). Using Axiom 3.8, construct the set $\set{x:Q(x)}$. Then $y\in\set{x:Q(x)}$ iff $(y=a$ or $y=b)$ which means that only elements of $\set{x:Q(x)}$ are $a$ and $b$ which we can denote $\set{a,b}$, the pair set formed by $a$ and $b$.\\
    \item[(3.4)] Let $A$ and $B$ be sets and let $P(x):x\in A \text{ or } x\in B$. Axiom 3.8 lets us construct the set $\set{x:P(x)}=\set{x:x\in A \text{ or } x\in B}$ which we denote $A\cup B$, the union of $A$ and $B$. \\
    \item[(3.5)] Let $A$ be a set. Let $Q(x):x\in A \text{ and } P(x)$ is true, where $P(x)$ is some other property pertaining to $x$. Using Axiom 3.8, we construct the set $\set{x:Q(x)}$ which has the property that for any object $y$, $y\in \set{x:Q(x)}$ iff $y\in A$ and $P(y)$ is true, which is exactly Axiom 3.5 and we are done. \\
    \item[(3.6)] 
    \item[(3.7)] Let \\
\end{enumerate}

\subsubsection{} Use the axiom of regularity (and the singleton set axiom) to show that if $A$ is a set, then $A\notin A$. Furthermore, show that if $A$ and $B$ are two sets, then either $A\notin B$ or $B\notin A$ (or both). \\
\soln Axiom 3.1 and 3.2 asserts that the set $A$ is an object and that we can construct the singleton set $\set{A}$. Since $\set{A}$ is a non-empty set, by Axiom 3.9, there is at least one element of $\set{A}$ which is disjoint from itself. In this case, the only element of $\set{A}$ is the set $A$ so $A\cap\set{A}=\emptyset$. Suppose on the contrary that $A\in A$, then $A\cap\set{A}=\set{A}\neq\emptyset$, a contradiction. Similarly, we can also construct the pair set $\set{A,B}$ by Axioms 3.1 and 3.2. Clearly, this set is non-empty so Axiom 3.9 asserts that either $A\cap\set{A,B}=\emptyset$ or $B\cap\set{A,B}=\emptyset$. Suppose on the contrary that $A\in B$ and $B\in A$, then $A\cap\set{A,B}=\set{B}\neq\emptyset$ or $B\cap\set{A,B}=\set{A}\neq\emptyset$. In either case, this contradicts Axiom 3.9 and we are done. \\


\subsubsection{} Show (assuming the other axioms of set theory) that the universal specification axiom, Axiom 3.8, is equivalent to an axiom postulating the existence of a "universal set" $\Omega$ consisting of all objects (i.e., for all objects $x$, we have $x\in\Omega$). In other words, if Axiom 3.8 is true, then a universal set exists, and conversely, if a universal set exists, then Axiom 3.8 is true. (This may explain why Axiom 3.8 is called the axiom of \textit{universal} specification). Note that if a universal set $\Omega$ existed, then we would have $\Omega\in\Omega$ by Axiom 3.1, contradicting Exercise 3.2.2. Thus the axiom of foundation specifically rules our the axiom of universal specification. \\
\soln Suppose the universal specification axiom, Axiom 3.8, holds. Then we can construct a set $\Omega=\set{x: x\text{ is \textit{any} object}}$. In other words, $x\in\Omega$ for any object $x$. Conversely, suppose a universal set $\Omega$ exists, a set containing all objects i.e., $x\in\Omega$ for any object $x$. Then, given any property $P(x)$ pertaining to an object $x$, we can construct a set $\set{x\in\Omega: P(x)\text{ is true}}$ by the axiom of specification and we have for every object $y$, $y\in\set{x\in\Omega: P(x)\text{ is true}}$ iff $y\in\Omega$ and $P(y)$ is true. However since $y\in\Omega$ is always true, we have $y\in\set{x\in\Omega: P(x)\text{ is true}}$ iff $P(y)$ is true, which is equivalent to Axiom 3.8. \\

\textit{Remark.} This section was \textit{fun}. \\


\subsection{Functions}

\subsubsection{} Show that the definition of equality in Definition 3.3.7 is reflexive, symmetric, and transitive. Also verifty the substitution property: if $f,\tilde{f}:X\to Y$ and $g,\tilde{g}:Y\to Z$ are functions such that $f=\tilde{f}$ and $g=\tilde{g}$, then $g\circ f=\tilde{g}\circ\tilde{f}$. \\
\soln 
\begin{enumerate}
\item[(a)] Reflexive: Given a function $f:X\to Y$ defined by $x\mapsto f(x)$, clearly $f$ has the same domain and codomain with itself and $f(x)=f(x)$ for all $x\in X$. Thus, $f=f$. 
\item[(b)] Symmetric: Let $f$ and $g$ be functions from $X\to Y$ such that $f=g$. Then, $f(x)=g(x)$ for all $x\in X$ and since this equality in the codomain is assumed to be well defined, then it is symmetric so $g(x)=f(x)$. Thus, $g=f$.
\item[(c)] Transitive: Let $f,g,h$ be functions from $X\to Y$ and suppose that $f=g$ and $g=h$. Then, $f(x)=g(x)$ and $g(x)=h(x)$ for all $x\in X$ and again, since equality in the codomain is assumed to be well defined, then it is also transitive so $f(x)=h(x)$ as well. Hence, $f=h$.
\item[(d)] Substitution: We have $f(x)=\tilde{f}(x)$ and $g(x)=\tilde{g}(x)$ for all $x\in X$. Since equality in the codomains $Y$ and $Z$ is assumed to be well defined, the substitution property must hold so $g\circ f(x)=g(f(x))=\tilde{g}(\tilde{f}(x))=\tilde{g}\circ\tilde{f}(x)$ for all $x\in X$. Clearly the domains and codomains are the same, that is $X\to Z$, so $g\circ f=\tilde{g}\circ\tilde{f}$. \\
\end{enumerate}

\subsubsection{} Let $f:X\to Y$ and $g:Y\to Z$ be functions. Show that if $f$ and $g$ are both injective, then so is $g\circ f$; similarly, show that if $f$ and $g$ are both surjective, then so is $g\circ f$. \\
\soln 
\begin{enumerate}
\item[(a)] Since $f$ is injective, for any $a,b\in X$, $a\neq b$ implies that $f(a)\neq f(b)$. Then since $g$ is also injective, for any $a',b'\in Y$, $a'\neq b'$ implies that $g(a')\neq g(b')$. Note that $f(a),f(b)\in Y$ so $f(a)\neq f(b)$ implies that $g\circ f(a) \neq g\circ f(b)$. Thus, $g\circ f$ is also injective. \\
\item[(b)] We know that $g\circ f$ is a function from $X$ to $Z$. Let $c\in Z$. Since $g$ is surjective, we know that there exists $b\in Y$ such that $g(b)=c$. Furthermore, since $f$ is surjective, there exists $a\in X$ such that $f(a)=b$. Thus, we can take $a\in X$ so that $(g\circ f)(a)=g(f(a))=g(b)=c$. Hence, $g\circ f$ is also surjective. \\
\end{enumerate}


\subsubsection{} When is the empty function injective? surjective? bijective? \\
\soln The empty function is always injective and this is vacuously true because $x,x'\in \emptyset$ is always false for any objects $x,x'$. Another way to think about it is that the empty function doesn't map any two distinct elements to the same element in the codomain (because it doesn't map any element in the first place). The empty function is surjective iff the codomain is equal to the empty set. This is also true vacuously because the premise $y\in\emptyset$ is false. Otherwise, if the codomain is non-empty, then there must at least be one element $y$ in it. However, there are no elements in $\emptyset$ which map to $y$ so the empty function cannot be surjective in this case. Consequently, the empty function is bijective iff the codomain is equal to the empty set. \\

\subsubsection{} In this section we give some cancellation laws for composition. Let $f:X\to Y$, $\tilde{f}:X\to Y$, $g:Y\to Z$, and $\tilde{g}:Y\to Z$ be functions. Show that if $g\circ f=g\circ\tilde{f}$ and $g$ is injective, then $f=\tilde{f}$. Is the same statement true if $g$ is not injective? Show that if $g\circ f=\tilde{g}\circ f$ and $f$ is surjective, then $g=\tilde{g}$. Is the same statement true if $f$ is not surjective? \\
\soln Suppose $g\circ f=g\circ\tilde{f}$ and that $g$ is injective. We have $(g\circ f)(x)=g(f(x))=g(\tilde{f}(x))=(g\circ\tilde{f})(x)$ for all $x\in X$. Since $g$ is injective, it follows that $f(x)=\tilde{f}(x)$. Note that their domains and codomains are equal. Thus, $f=\tilde{f}$. We now show that left cancellation law doesn't hold if $g$ is not injective by giving a counterexample. Consider the functions $g:\R\to\R$ defined by $x\mapsto |x|$, $f:\R\to\R$ defined by $x\mapsto x$, and $\tilde{f}:\R\to\R$ defined by $x\mapsto -x$. Clearly, $g$ is not injective because both $x$ and $-x$ map to $x$. Observe that 
\[
(g\circ f)(x)=g(f(x))=g(x)=|x|=|-x|=g(-x)=g(\tilde{f}(x))=(g\circ \tilde{f})(x)
\]
and since the domains and codomains are equal, we have $g\circ f=g\circ\tilde{f}$. However, $f\neq \tilde{f}$ because $f(x)=x\neq -x=\tilde{f}(x)$ whenever $x\neq 0$. \\

Now, suppose that $g\circ f=g\circ\tilde{f}$ and that $f$ is surjective. We have $(g\circ f)(x)=g(f(x))=g(\tilde{f}(x))=(g\circ\tilde{f})(x)$ for all $x\in X$. Let $b\in Y$. Since $f$ is surjective, there exists an $a\in X$ such that $f(a)=b$. Substituting $f(a)=b$, we get $g(b)=\tilde{g}(b)$ for all $b\in Y$. Note that their domains and codomains are also equal. Hence, $g=\tilde{g}$. We also show that right cancellation law doesn't hold if $f$ is not surjective. Consider the functions $f,g,\tilde{g}$ from $\R$ to $\R$ defined by $f(x)=0$, $g(x)=e^x$, and $\tilde{g}(x)=x+1$. Clearly, $f$ is not surjective because, for example, there doesn't exist an $a\in\R$ such that $f(a)=1$. We have 
\[
(g\circ f)(x)=g(f(x))=g(0)=e^0=1=0+1=\tilde{g}(0)=\tilde{g}(f(x))=\tilde{g}\circ f(x)
\]
and since their respective domains and codomains are equal, $g\circ f=\tilde{g}\circ f$. However, it should be obvious that $g\neq \tilde{g}$. \\

\subsubsection{} Let $f:X\to Y$ and $g:Y\to Z$ be functions. Show that if $g\circ f$ is injective, then $f$ must be injective. Is it true that $g$ must also be injective? Show that if $g\circ f$ is surjective, then $g$ must be surjective. Is it true that $f$ must also be surjective? \\
\soln 
\begin{enumerate}
\item[(a)] We will prove the contrapositive. Suppose that $f$ is \textit{not} injective, then there exists $x,x'\in X$ such that $x\neq x'$ but $f(x)=f(x')$. It follows that $g(f(x))=g(f(x'))$. Thus, $g\circ f$ is not injective either. We give a counterexample to show that $g\circ f$ being injective does not imply that $g$ is injective, in general. Consider the functions $f:[0,\infty)\to\R$ and $g:\R\to [0,\infty)$ defined by $f(x)=x$ and $g(x)=x^2$. Observe that if $a\neq b$, then $(g\circ f)(a)=a^2\neq b^2=(g\circ f)(b)$. This can be shown by contradiction by noting that the equation $a^2-b^2=(a+b)(a-b)=0$ is true iff $a=b$ or $a=-b$ but both $a,b\in [0,\infty)$ so $a=-b$ is negative whenever they are non-zero leaving us only with $a=b$, a contradiction. Thus, $g\circ f$ is injective even though $g$ is not injective. \\
\item[(b)] Suppose $g$ is not surjective, then there exists $b\in Z$ such that for all $a\in Y$, we have $g(a)\neq b$. Note that for all $x\in X$, $f(x)\in Y$ as well so $(g\circ f)(x)=g(f(x))\neq b$. Thus, $g\circ f$ is not surjective either. We now show that the surjectivity of $g\circ f$ does not imply that $f$ is also surjective, in general. Consider $f:\R\to\R$ where $f(x)=\sin x$ and $g:\R\to\set{-1,0,1}$ where $g(x)=\mathop{\rm sgn}x$. Then $g\circ f$ is surjective but $f$ is not. \\
\end{enumerate}

\subsubsection{} Let $f:X\to Y$ be a bijective function, and let $f^{-1}:Y\to X$ be its inverse. Verify the cancellation law $f^{-1}(f(x))=x$ for all $x\in X$ and $f(f^{-1}(y))$ for all $y\in Y$. Conclude that $f^{-1}$ is also invertible, and has $f$ as its inverse (thus $(f^{-1})^{-1}=f$). \\
\soln Since $f$ is bijective, for every $y\in Y$, there is exactly one $x\in X$ such that $f(x)=y$. By definition, this unique $x$ is denoted $x=f^{-1}(y)$. Analogously, by definition of a function, for every $x\in X$, there is exactly one $y\in Y$ such that $f(x)=y$. Hence, $f^{-1}(f(x))=f^{-1}(y)=x$ and $f(f^{-1}(y))=f(x)=y$. Next, we show that $f^{-1}$ is a well-defined function and also invertible or bijective;
\begin{enumerate}
\item[(a)] \textit{well-defined}: This follows immediately from the bijectivity of $f$, i.e., for every $y\in Y$, there is a unique $x\in X$ such that $f^{-1}(y)=x$. 
\item[(b)] \textit{injective}: Let $y,y'\in Y$ where $y\neq y'$. Since $f$ is bijective, there is a unique $x\in X$ and $x'\in X$ such that $f(x)=y$ and $f(x')=y'$, which we denote respectively as $x=f^{-1}(y)$ and $x'=f^{-1}(y')$. Since $f$ is a function, it follows that $y=f(x)\neq f(x')=y'$ implies $x\neq x'$. Thus, $f^{-1}(y)\neq f^{-1}(y')$. Hence, $f^{-1}$ is injective
\item[(c)] \textit{surjective}: Let $x\in X$. Since $f$ is a function, there exists a unique $y\in Y$ such that $f(x)=y$ and the bijectivity of $f$ guarantees that for every $y$, this choice of $x$ is unique and thus we can write $f^{-1}(y)=x$. Hence, $f^{-1}$ is surjective. 
\end{enumerate}
We can now conclude that $f^{-1}$ is also bijective. Lastly, we want to show that $(f^{-1})^{-1}=f$. Note that since $f^{-1}$ is a function from $Y$ to $X$, $(f^{-1})^{-1}$ is a function from $X$ to $Y$ so the domain and codomain match with $f$. Furthermore, by the above arguments, we can conclude that $(f^{-1})^{-1}$ is bijective because $f^{-1}$ itself is bijective. We now apply the definition of bijectivity twice. Since $f$ is bijective, there is a unique $x\in X$ for each $y\in Y$ such that $f(x)=y$ and since $f^{-1}$ is bijective, there is a unique $y\in Y$ for each $x\in X$ such that $f^{-1}(y)=x$. For the latter, we denote the unique $y$ as $y=(f^{-1})^{-1}(x)$. Combining these, we have $f(x)=(f^{-1})^{-1}(x)$ for an arbitrary $x\in X$ and hence, $(f^{-1})^{-1}=f$ as desired. \\

\subsubsection{} Let $f:X\to Y$ and $g:Y\to Z$ be functions. Show that if $f$ and $g$ are bijective, then so is $g\circ f$, and we have $(g\circ f)^{-1}=f^{-1}\circ g^{-1}$. \\
\soln Suppose $f$ and $g$ are bijective. Note that $g\circ f$ is a function from $X$ to $Z$. It follows that for every $z\in Z$, we can choose a unique $y\in Y$ such that $g(y)=z$ because $g$ is bijective and furthermore, we can also choose a unique $x\in X$ such that $f(x)=y$ since $f$ is bijective. Then, $(g\circ f)(x)=g(f(x))=g(y)=z$ and hence, $g\circ f$ is bijective. \\

Next we show the sock-shoe principle, i.e., $(g\circ f)^{-1}=f^{-1}\circ g^{-1}$. We first verify that their domains and codomains are equal. We have $g\circ f:X\to Z$ so $(g\circ f)^{-1}:Z\to X$. On the other hand, we have $g^{-1}:Z\to Y$ and $f^{-1}:Y\to X$ so $f^{-1}\circ g^{-1}:Z\to X$. By Exercise 3.3.6, both $f^{-1}$ and $g^{-1}$ are bijective. Let $z\in Z$. Note that we have proven that $g\circ f$ is bijective so by Exercise 3.3.6, we have the cancellation law,
\[
((g\circ f)\circ(g\circ f)^{-1})(z)=z
\]
Since $g$ is bijective, there exists a unique $y\in Y$ such that $g(y)=z$ which we denote $y=g^{-1}(z)$. Recall that function composition is associative. Then using the cancellation law in Exercise 3.3.6 again, we have
\[
((g\circ f)\circ f^{-1}\circ g^{-1})(z)=g(f(f^{-1}(g^{-1}(z))))=g(f(f^{-1})(y))=g(y)=z
\]
Thus,
\[
(g\circ f)\circ(g\circ f)^{-1}=(g\circ f)\circ f^{-1}\circ g^{-1}
\]
and by Exercise 3.3.4, we can use the left cancellation law for composition because $g\circ f$ is injective (since it is bijective). Hence, $(g\circ f)^{-1}=f^{-1}\circ g^{-1}$ as desired. \\


\subsubsection{} If $X$ is a subset of $Y$, let $\iota_{X\to Y}:X\to Y$ be the \textit{inclusion map  from $X$ to $Y$}, defined by mapping $x\mapsto x$ for all $x\in X$, i.e., $\iota_{X\to Y}(x):=x$ for all $x\in X$. The map $\iota_{X\to X}$ is in particular called the \textit{identity map} on $X$. 
\begin{enumerate}
\item[(a)] Show that if $X\subseteq Y\subseteq Z$ then $\iota_{Y\to Z}\circ\iota_{X\to Y}=\iota_{X\to Z}$. 
\item[(b)] Show that if $f:A\to B$ is any function, then $f=f\circ\iota_{A\to A}=\iota_{B\to B}\circ f$.
\item[(c)] Show that, if $f:A\to B$ is a bijective function, then $f\circ f^{-1}=\iota_{B\to B}$ and $f^{-1}\circ f=\iota_{A\to A}$. 
\item[(d)] Show that if $X$ and $Y$ are disjoint sets, and $f:X\to Z$ and $g:Y\to Z$ are functions, then there is a unique function $h:X\cup Y\to Z$ such that $h\circ \iota_{X\to X\cup Y}=f$ and $h\circ \iota_{Y\to X\cup Y}=g$. \\
\end{enumerate}

\textit{Solution.} 
\begin{enumerate}
\item[(a)] Note that $\iota_{Y\to Z}\circ\iota_{X\to Y}$ and $\iota_{X\to Z}$ are both functions from $X$ to $Z$. Furthermore it should be clear that 
\[
\qquad\quad (\iota_{Y\to Z}\circ\iota_{X\to Y})(x)=\iota_{Y\to Z}(\iota_{X\to Y}(x))=\iota_{Y\to Z}(x)=x=\iota_{X\to Z}(x)
\]
for all $x\in X$. Hence, $\iota_{Y\to Z}\circ\iota_{X\to Y}=\iota_{X\to Z}$.
\item[(b)] We have $f\circ\iota_{A\to A}:A\to B$ and $\iota_{B\to B}\circ f:A\to B$. Observe that for all $x\in A$, we have
\[
(f\circ\iota_{A\to A})(a)=f(\iota_{A\to A}(a))=f(a)
\]
and since $f(x)\in B$,
\[
(\iota_{B\to B}\circ f)(x)=\iota_{B\to B}(f(x))=f(x).
\]
Therefore, $f=f\circ\iota_{A\to A}=\iota_{B\to B}\circ f$.
\item[(c)] We have $f\circ f^{-1}:B\to B$ and $f^{-1}\circ f:A\to A$. We will use the cancellation laws for composition of inverse functions in Exercise 3.3.6. Let $b\in B$, then
\[
(f\circ f^{-1})(b)=f(f^{-1})(b)=b=\iota_{B\to B}(b).
\]
Similarly, let $a\in A$, then
\[
(f^{-1}\circ f)(a)=f^{-1}(f(a))=a=\iota_{A\to A}(a).
\]
Thus, $f\circ f^{-1}=\iota_{B\to B}$ and $f^{-1}\circ f=\iota_{A\to A}$.
\item[(d)] 



\end{enumerate}


\subsection{Images and inverse images} 

\subsubsection{} Let $f:X\to Y$ be a bijective function, and let $f^{-1}:Y\to X$ be its inverse. Let $V$ be any subset of $Y$. Prove that the forward image of $V$ under $f^{-1}$ is the same as the inverse image of $V$ under $f$; thus the fact that both sets are denoted $f^{-1}(V)$ will not lead to any inconsistency. \\

\textit{Solution.} We have the forward image of $V$ under $f^{-1}$, $A=\set{f^{-1}(y):y\in V}$, and the inverse image of $V$ under $f$, $B=\set{x\in X: f(x)\in V}$. Suppose $x\in A$, then $x=f^{-1}(y)$ for some $y\in V$ and since $f^{-1}$ is bijective (because $f$ is bijective), this $y$ is unique. Using the cancellation law for bijective functions, we have $f(x)=f(f^{-1}(y))=y$ so $f(x)\in V$. Thus, $x\in B$ which follows that $A\subseteq B$. Suppose now that $x\in B$, then $f(x)\in V$. Let $y=f(x)$. Using again the cancellation law, we have $f^{-1}(y)=f^{-1}(f(x))=x$ and so $x\in A$ which follows that $B\subseteq A$. Hence, $A=B$ as desired. \\


\subsubsection{} Let $f:X\to Y$ be a function from one set $X$ to another set $Y$, let $S$ be a subset of $X$, and let $U$ be a subset of $Y$. What, in general, can one say about $f^{-1}(f(S))$ and $S$? What about $f(f^{-1}(U))$ and $U$? \\

\textit{Solution.} Observe that $x\in S \implies f(x)\in f(S) \iff x\in f^{-1}(f(S))$ so in general, $S\subseteq f^{-1}(f(S))$.  On the other hand, $x\in f^{-1}(f(S)) \iff f(x)\in f(S)$, however $f(x)\in f(S)\centernot\implies x\in S$ (a counterexample was shown in the text) so in general, $f^{-1}(f(S))\centernot\subseteq S$. The special case is when $f$ is injective because then, the uniqueness of $x$ is guaranteed and we can conclude $f(x)\in f(S)\implies x\in S$ and thus, $S=f^{-1}(f(S))$ \\

Suppose now that $y\in f(f^{-1}(U))$ then it follows that $y=f(x)$ for some $x\in f^{-1}(U)$. Note that for any particular $x\in X$, there is only one such $y\in Y$ such that $f(x)=y$ (this is part of the definition of a function). Since $x\in f^{-1}(U) \iff f(x)\in U$, we conclude that $y\in U$ so $f(f^{-1}(U))\subseteq U$. However, if we suppose that $y\in U$ we cannot, in general, conclude that $y=f(x)$ for some $x\in X$ (we can only do so when $f$ is surjective) and in turn, also that $U\subseteq f(f^{-1}(U))$. As a counterexample, consider the sets $X=\set{2,3}$, $Y=\set{4,9,16}$, and $U=\set{4,16}$ and the squaring function $f:X\to Y$, $x\mapsto x^2$. We have $f(f^{-1}(U))=f(\set{2})=\set{4}$ and clearly $U=\set{4,16}\neq\set{4}=f(f^{-1}(U))$. Thus, in general, $U\centernot\subseteq f(f^{-1}(U))$. \\


\subsubsection{} Let $A,B$ be two subsets of a set $X$, and let $f:X\to Y$ be a function. Show that $f(A\cap B)\subseteq f(A)\cap f(B)$, that $f(A)\setminus f(B)\subseteq f(A\setminus B)$, $f(A\cup B)=f(A)\cup f(B)$. For the first two statements, is it true that the $\subseteq$ relation can be improved to $=$? \\

\textit{Solution.} Recall that $y\in f(S) \iff y=f(x)$ for some $x\in S$.
\begin{enumerate}
\item Let $y\in f(A\cap B)$. Then $y=f(x)$ for some $x\in A\cap B$. By definition of set intersection, we have $x\in A$ and $x\in B$. It then follows that $y\in f(A)$ and $y\in f(B)$, that is, $y\in f(A)\cap f(B)$. Hence, $f(A\cap B)\subseteq f(A)\cap f(B)$. \\

\item Let $y\in f(A)\setminus f(B)$. Then $y\in f(A)$ and $y\notin f(B)$. Since $y\in f(A)$, $y=f(x)$ for some $x\in A$. On the other hand, since $y\notin f(B)$, $y\neq f(x)$ for any $x\in B$. It follows that $x\in A\setminus B$ and so $y\in f(A\setminus B)$. Hence, $f(A)\setminus f(B)\subseteq f(A\setminus B)$. \\

\item Let $y\in f(A\cup B)$. Then $y=f(x)$ for some $x\in A\cup B$. By definition of set union, we have $x\in A$ or $x\in B$. Suppose WLOG that $x\notin B$, then $x\in A$ and thus $y\in f(A)$. Hence, $y\in f(A)\cup f(B)$ and so $f(A\cup B)\subseteq f(A)\cup f(B)$. Conversely, suppose that $y\in f(A)\cup f(B)$. Then $y\in f(A)$ or $y\in f(B)$. Suppose again WLOG that $y\notin f(B)$, then $y\in f(A)$ and so $y=f(x)$ for some $x\in A$. Note that $A\subseteq A\cup B$ so $x\in A\cup B$. Thus, $y\in f(A\cup B)$ and so $f(A)\cup f(B)\subseteq f(A\cup B)$. By mutual inclusion, we conclude that $f(A\cup B)=f(A)\cup f(B)$. \\
\end{enumerate}

The answer to the last sentence is \textbf{no}---which we will show by giving counterexamples. For (1), consider the sets $X=Y=\set{1,2,3}$, $A=\set{1,2}$, and $B=\set{1,3}$ and the function $f:$ $1\mapsto 2$, $2\mapsto 3$, and $3\mapsto 3$. Then, $f(A\cup B)=\set{2}$ and $f(A)\cup f(B)=\set{2,3}$ but $\set{2}\neq\set{2,3}$. For (2), consider sets $X=Y=A=\set{a,b}$ and $B=\set{b}$ and the function $f:$ $a\mapsto b$, $b\mapsto b$. Then $f(A)\setminus f(B)=\emptyset$ and $f(A\setminus B)=\set{b}$ but $\emptyset\neq\set{b}$. \\


\subsubsection{} Let $f:X\to Y$ be a function from one set $X$ to another set $Y$ and let $U,V$ be subsets of $Y$. Show that $f^{-1}(U\cup V)=f^{-1}(U)\cup f^{-1}(V)$, that $f^{-1}(U\cap V)=f^{-1}(U)\cap f^{-1}(V)$, and that $f^{-1}(U\setminus V)=f^{-1}(U)\setminus f^{-1}(V)$. \\

\textit{Solution.} 
\begin{enumerate}
\item Let $x\in f^{-1}(U\cup V)$. Then $f(x)\in U\cup V \iff f(x)\in U$ or $f(x)\in V$. Suppose $f(x)\notin V$, then $f(x)\in U$ which means that $x\in f^{-1}(U)\subseteq f^{-1}(U)\cup f^{-1}(V)$. Thus, $f^{-1}(U\cup V)\subseteq f^{-1}(U)\cup f^{-1}(V)$. Conversely, let $x\in f^{-1}(U)\cup f^{-1}(V)$, then $x\in f^{-1}(U)$ or $x\in f^{-1}(V)$. Suppose $x\notin f^{-1}(V)$, then $x\in f^{-1}(U) \iff f(x)\in U$. Note that $U\subseteq U\cup V$ so $f(x)\in U\cup V\iff x\in f^{-1}(U\cup V)$. Thus, $f^{-1}(U)\cup f^{-1}(V)\subseteq f^{-1}(U\cup V)$ and hence, $f^{-1}(U\cup V)=f^{-1}(U)\cup f^{-1}(V)$. \\

\item Let $x\in f^{-1}(U\cap V)$, then
\begin{align*}
    f(x)\in U\cap V &\iff f(x)\in U \text{ and } f(x)\in V \\
    &\iff x\in f^{-1}(U) \text{ and } x\in f^{-1}(V) \\
    &\iff x\in f^{-1}(U)\cap f^{-1}(V) \\
    &\,\implies f^{-1}(U\cap V)\subseteq f^{-1}(U)\cap f^{-1}(V). 
\end{align*}
Conversely, let $x\in f^{-1}(U)\cap f^{-1}(V)$, then
\begin{align*}
    x\in f^{-1}(U) \text{ and } x\in f^{-1}(V) &\iff f(x)\in U \text{ and } f(x)\in V \\
    &\iff f(x)\in U\cap V \\
    &\iff x\in f^{-1}(U\cap V) \\
    &\,\implies f^{-1}(U)\cap f^{-1}(V)\subseteq f^{-1}(U\cap V).
\end{align*}
And so we conclude that $f^{-1}(U\cap V)=f^{-1}(U)\cap f^{-1}(V)$. \\

\item Let $x\in f^{-1}(U\setminus V)$, then
\begin{align*}
    f(x)\in U\setminus V &\iff f(x)\in U \text { and } f(x)\notin V \\
    &\iff x\in f^{-1}(U) \text{ and } x\notin f^{-1}(V) \\
    &\iff x\in f^{-1}(U)\setminus f^{-1}(V) \\
    &\,\implies f^{-1}(U\setminus V)\subseteq f^{-1}(U)\setminus f^{-1}(V). 
\end{align*}
Conversely, let $x\in f^{-1}(U)\setminus f^{-1}(V)$, then 
\begin{align*}
    x\in f^{-1}(U) \text{ and } x\notin f^{-1}(V) &\iff f(x)\in U \text{ and } f(x)\notin V \\
    &\iff f(x)\in U\setminus V \\
    &\iff x\in f^{-1}(U\setminus V) \\
    &\,\implies f^{-1}(U)\setminus f^{-1}(V)\subseteq f^{-1}(U\setminus V).
\end{align*}
And so we conclude $f^{-1}(U\setminus V)=f^{-1}(U)\setminus f^{-1}(V)$. \\
\end{enumerate}


\subsubsection{}  Let $f:X\to Y$ be a function from one set $X$ to another set $Y$. Show that $f(f^{-1}(S))=S$ for every $S\subseteq Y$ if and only if $f$ is surjective. Show that $f^{-1}(f(S))=S$ for every $S\subseteq X$ if and only if $f$ is injective. \\

\textit{Solution.}
\begin{enumerate} 
\item If $f$ is surjective, then for any $y\in Y$, there exists $x\in X$ such that $f(x)=y$. In other words, $f(X)=Y$. From Exercise 3.4.2, we already know that $f(f^{-1}(S))\subseteq S$. Suppose now that $y\in S$, then because $S\subseteq Y$, we also know that for any $y\in S$, there exists $x\in X$ such that $f(x)=y\in S$. By definition of inverse images, it follows that $x\in f^{-1}(S)$. Finally, by definition of forward images, we have $f(x)=y\in f(f^{-1}(S))$. Hence, $f(f^{-1}(S))=S$. \\

\item If $f$ is injective, then for any $x,x'\in X$ we have $f(x)=f(x')\implies x=x'$. In other words, the uniqueness of $x$ as a pre-image is guaranteed. From Exercise 3.4.2, we already know that $S\subseteq f^{-1}(f(S))$. Suppose now that $x\in f^{-1}(f(S))$, then $f(x)\in f(S)$. Suppose on the contrary that $x\notin S$, there must be another $x'\neq x$ such that $f(x')=f(x)$. However, this implies that $x=x'$ by the injectivity of $f$, a contradiction. Thus we conclude that $x\in S$ and so $f^{-1}(f(S))=S$. \\
\end{enumerate}


\subsubsection{} Prove Lemma 3.4.9. (Hint: start with the set $\set{0,1}^X$ and apply the replacement axiom, replacing each function $f$ with the object $f^{-1}(\set{1})$.) See also Exercise 3.5.11.







%%%%%%%%%%%%%
%Integers and rationals

\newpage
\section{Integers and rationals}

\subsection{The integers} 

\subsubsection{}
Verify that the definition of equality on the integers is both reflexive and symmetric. \\
\soln Suppose $a\mns b$ is an integer. We know that $a$ and $b$ are natural numbers so $a+b=a+b$ which follows that $a\mns b=a\mns b$ by definition of equality of integers. Next, suppose $a\mns b=c\mns d$, then we have $a+d=c+b$. Note that we have defined equality on the natural numbers to be symmetric so $c+b=a+d$ and thus, $c\mns d=a\mns b$. Hence, we are done. \\
 
\subsubsection{} Show that the definition of negation on the integers is well-defined in the sense that if $(a\mns b)=(a'\mns b')$, then $-(a\mns b)=-(a'\mns b')$ (so equal integers have equal negations). \\
\soln Let $(a\mns b)=(a'\mns b')$, then $a+b'=a'+b$. By definition of negation (Definition 4.1.4), we have $-(a\mns b)=b\mns a$ and $-(a'\mns b')=b'\mns a'$. Observe that $a+b'=a'+b$ implies that $b'+a=b+a'$ because addition of natural numbers is commutative (Proposition 2.2.4). It then follows from Definition 4.1.1 that $b'\mns a'=b\mns a$ and since we have already shown that equality on integers is symmetric, we have $b\mns a=b'\mns a'$ and thus, $-(a\mns b)=-(a'\mns b')$. \\
 
\subsubsection{} Show that $(-1)\times a=-a$ for every integer $a$. \\
\soln Let $a=b\mns c$. Note that $1=1\mns0$ so $-1=0\mns 1$. We have 
\[
(-1)\times a=(0\mns 1)\times (b\mns c)=(0b+1c)\mns (0c+1b)=c\mns b=-a
\]
as desired. \\
 
\subsubsection{} Prove the remaining identities in Proposition 4.1.6. (Hint: one can save some work by using some identities to prove others. For instance, once you know that $xy=yx$, you get for free $x1=1x$, and once you also prove $x(y+z)=xy+xz$, you automatically get $(y+z)x=yx+zx$ for free.) \\
\soln Let $x=a\mns b$, $y=c\mns d$, and $z=e\mns f$. We will use the fact that addition and multiplication of natural numbers is commutative and associative. First, for commutativity of addition:
\begin{align*}
x+y=(a\mns b)+(c\mns d)&=(a+c)\mns (b+d)\\
&=(c+a)\mns (d+b)=(c\mns d)+(a\mns b)=y+x
\end{align*}
Next, we have associativity of addition:
\begin{align*}
(x+y)+z&=((a\mns b)+(c\mns d))+(e\mns f)\\
&=((a+c)\mns(b+d))+(e\mns f) \\
&=((a+c)+e)\mns((b+d)+f) \\
&=(a+(c+e))\mns(b+(d+f)) \\
&=(a\mns b)+((c+e)\mns(d+f)) \\
&=(a\mns b)+((c\mns d)+(e\mns f))=x+(y+z)
\end{align*}
Then for the additive identity:
\[
x+0=(a\mns b)+(0\mns 0)=(a+0)\mns (b+0)=a\mns b=x
\]
Since we've already shown that addition of integers is commutative, we also have $x+0=0+x=x$. Next, we have the additive inverse:
\[
x+(-x)=(a\mns b)+(b\mns a)=(a+b)\mns (a+b)=0
\]
 and again, by commutativity of addition, $x+(-x)=(-x)+x=0$. Associativity of addition has already been shown in the text. Next, we have commutativity of multiplication of integers: 
\begin{align*}
xy=(a\mns b)(c\mns d)&=(ac+bd)\mns(ad+bc)\\
&=(ca+db)\mns(cb+da)=(c\mns d)(a\mns b)=yx
\end{align*}
Associativity of multiplication has already been shown. Next, we have the multiplicative identity: 
\[
x1=(a\mns b)(1\mns 0)=(a1+b0)\mns(a0+b1)=a\mns b=x
\]
and since we have already shown commutativity, we also have $x1=1x=x$. Lastly, we have distributivity of multiplication over addition:
\begin{align*}
x(y+z)&=(a\mns b)((c\mns d)+(e\mns f))\\
&=(a\mns b)((c+e)\mns (d+f)) \\
&=(a(c+e)+b(d+f))\mns (a(d+f)+b(c+e)) \\
&=(ac+ae+bd+bf)\mns(ad+af+bc+be) \\
&=((ac+bd)+(ae+bf)))\mns((ad+bc)+(af+be)) \\
&=((ac+bd)\mns(ad+bc))+((ae+bf)\mns(af+be)) \\
&=(a\mns b)(c\mns d)+(a\mns b)(e\mns f) \\
&=xy+xz
\end{align*}
Then for the last one, by commutativity of multiplication we have:
\[
(y+z)x=x(y+z)=xy+xz=yx+zx
\]
and we are done. \\


\subsubsection{} Prove Proposition 4.1.8. (Hint: while this proposition is not quite the same as Lemma 2.3.3, it is certainly legitimate to use Lemma 2.3.3 in the course of proving Proposition 4.1.8.) \\
\soln Let $a,b$ be integers such that $ab=0$. We will use the fact that $-a=(-1)\times a$ for any integer $a$ from Exercise 4.1.3. Suppose on the contrary that $a\neq 0$ and $b\neq 0$ We split this into three cases:
\begin{enumerate}
\item[(a)] If both $a$ and $b$ are positive: Then by Lemma 2.3.3, $ab$ is also positive which is a contradiction. 
\item[(b)] If both $a$ and $b$ are negative: Then $-a$ and $-b$ are positive and we have $(-a)(-b)=(-1)(a)(-1)(b)=ab$ which, by Lemma 2.3.3, is also positive, a contradiction. 
\item[(c)] If $a$ is negative and $b$ is positive: Then $-a$ is positive and we have $(-a)(b)=(-1)(a)(b)=-ab$ which, by Lemma 2.3.3, is positive. Thus $ab$ is negative which is also a contradiction to our assumption.
\end{enumerate}
In all cases, we reach a contradiction so we must have $a=0$ or $b=0$, as desired. \\

\subsubsection{} Prove Corollary 4.1.9. (Hint: there are two ways to do this. One is to use Proposition 4.1.8 to conclude that $a-b$ must be zero. Another way is to combine Corollary 2.3.7 with Lemma 4.1.5.) \\
\soln Suppose $ac=bc$ where $a,b,c$ are integers and $c\neq 0$. We have $ac-bc=(a-b)c=0$. Since $c\neq 0$, we must have $a-b=0$ by Proposition 4.1.8 and hence, $a=b$ as desired.



\subsubsection{} Prove Lemma 4.1.11. (Hint: use the first part of this lemma to prove all the others.) \\
\soln 
\begin{enumerate}
\item[(a)] $(\implies)$ Suppose $a>b$, then $a\geq b$ and $a\neq b$. By Definition $4.1.10$, we can write $a=b+n$ for some natural number $n$. If we suppose that $n=0$, then $a=b$ which is a contradiction so $n$ is a strictly positive natural number. Subtracting $b$ from both sides we get $a-b=n$ and we are done for the forward direction. $(\impliedby)$ Suppose $a-b$ is a positive natural number, then $a-b=m$ for some positive natural number $m$. We are sure that $a\neq b$ because otherwise, we would have $m=0$, a contradiction. Adding $b$ to both sides, we get $a=b+m$ so $a\geq b$. Thus, $a>b$. \\
\item[(b)] Suppose $a>b$, then $a-b$ is a positive natural number by (a). Note that for any integer $c$, $c-c=0$ so $a-b=(a-b)+0=(a-b)+(c-c)=(a+c)-(b+c)$. In other words, $(a+c)-(b+c)$ is also a positive natural number so $a+c>b+c$. \\
\item[(c)] Suppose $a>b$ and $c$ is positive. Then $a-b=n$ for some positive natural number $n$. Multiplying both sides by $c$ gives $ac-bc=nc$. Since $n$ is positive and $c$ is positive, $nc$ is positive which means $ac-bc$ is also positive (because they are equal). Hence, by (a), we have $ac>bc$ as desired.
\item[(d)] Suppose $a>b$, then $a-b$ is a positive natural number. Observe that $-(a-b)=(-a)-(-b)=b-a$ is a negative integer. We can write $b-a=-n$ where $n$ is a positive natural number. Then adding $-b$ and $n$ to both sides, we get $-b=-a+n$. Thus $-b>-a$ and equivalently $-a<-b$.
\item[(e)] Suppose $a>b$ and $b>c$, then by (a), $a-b$ and $b-c$ are positive natural numbers. Proposition 2.2.8 asserts that $(a-b)+(b-c)=a-c$ is also positive. Thus, by (a) again, $a>c$.
\item[(f)] Let $a$ and $b$ be integers. Note that $a-b$ is also an integer. By Lemma 4.1.5, exactly one of the following is true: (A) $a-b=0$, (B) $a-b$ is positive, or (C) $b-a$ is positive. If (A) is true then $a=b$. If (B) is true then $a>b$ by (a). If (C) is true then $b>a$ and equivalently, $a<b$ and we are done. \\
\end{enumerate}


\subsubsection{} Show that the principle of induction (Axiom 2.5) does not apply directly to the integers. More precisely, give an example of a property $P(n)$ pertaining to an integer $n$ such that $P(0)$ is true, and that $P(n)$ implies $P(n\pls)$ for all integers $n$, but that $P(n)$ is not true for all integers $n$. Thus, induction is not as useful as tool for dealing with the integers as it is with the natural numbers. (The situation becomes even worse with the rational and real numbers, which we shall define shortly.) \\
\soln Let $P(n)$ be the property that $n\geq 0$ where $n$ is an integer. Clearly, $P(0)$ is true because $0\geq 0$. Suppose inductively that $P(n)$ holds. We have $n\geq 0$ by the induction hypothesis and we know that $n\pls\geq n$ so $n\pls \geq 0$. Hence, $P(n\pls)$ holds whenever $P(n)$ is true. However, $P(n)$ is not true for all integers. Indeed, it is not true whenever $n$ is a negative integer. If we suppose otherwise, we have $n=0\mns a$ for some positive natural number $a$ because $n$ is a negative integer. On the other hand, $P(n)$ holds so $n\geq 0$, meaning there exists some positive natural number $b$ such that $n=0+b=b=b\mns 0$. It follows that $0\mns a=b\mns 0$ which by Definition 4.1.1, implies that $a+b=0$. This is a contradiction because the sum of two positive natural numbers is also a positive natural number (Proposition 2.2.8). The principle of induction fails to apply to the integers because the integers are not well-ordered, i.e., the integers do not have a least element. \\

\textit{Remark.} The domino analogy helps in visualizing this. The base case is the domino which we will push to the right. This domino knocks down all of the infinitely many dominoes to its right but there are still an infinite amount of dominoes to its left that are standing. \\


\subsection{The rationals}

\subsubsection{} Show that the definition of equality for the rational numbers is reflexive, symmetric, and transitive. (Hint: for transitivity, use Corollary 4.1.9.) \\
\soln We will show that equality on the rationals inherits reflexivity, symmetry, and transitivity from the integers. (Reflexive) For any integers $a$ and $b$ where $b$ is non-zero, we have $ab=ab$ since equality on the integers is reflexive. It follows from Definition 4.2.1 that $a\dvd b=a\dvd b$. (Symmetric) Suppose $a//b=c//d$, where $a//b, c//d$ are rationals, then by definition we have $ad=cb$ which follows that $cb=ad$ because equality on the integers is symmetric. Thus, $c//d=a//b$. (Transitive) Suppose $a//b=c//d$ and $c//d=e//f$ where $a//b,\ c//d,\ e//f$ are rationals. By Definition 4.2.1, we have $ad=cb$ and $cf=ed$. Multiplying the two equations together, we get $adcf=cbed$. By Corollary 4.1.9, we can cancel $c$ and $d$ to obtain $af=eb$ and thus, $a//b=e//f$ and we are done. \qed \\


\subsubsection{} Prove the remaining components of Lemma 4.2.3. \\
\soln (Product) Suppose $a//b=a'//b'$, then $ab'=a'b$ where $b,b'$ are non-zero. We want to show that $(a//b)(c//d)=(a'//b')(c//d)$---it suffices to show this for only the left factor because multiplication is commutative. Note that $d$ is also non-zero so $bd$ and $b'd$ are non-zero as well by Proposition 4.1.8. By definition of multiplication of rationals, we have $(ac)//(bd)$ for the left hand side and $(a'c)//(b'd)$ for the right hand side. We then have to show that $(ac)(b'd)=(a'c)(bd)$ or that $ab'cd=a'bcd$ (because multiplication of integers is commutative and associative) but since $ab'=a'b$, we know that this is true. \\

(Negation) Suppose $a//b=a'//b'$, then $ab'=a'b$. We want to show that $-(a//b)=-(a'//b')$, i.e., equal rationals have equal negations. By definition, we have $(-a)//b$ for the left hand side and $(-a')//b'$ for the right hand side so we want to show that $(-a)b'=(-a')b$ which is the same as $-(ab')=-(a'b)$ but since $ab'=ab'$ and negation on integers is well-defined, the claim follows. \qed \\

\subsubsection{} Prove the remaining components of Proposition 4.2.4. (Hint: as with Proposition 4.1.6, you can save some work by using some identities to prove others.) \\
\soln Let $x=a//b$, $y=c//d$, and $z=e//f$ for integers $a,c,e$ and non-zero integers $b,d,f$. For commutativity of addition and multiplication, observe that
\begin{align*}
x+y=(a//b)+(c//d)&=(ad+bc)//bd\\
&=(cb+da)//db=(c//d)+(a//b)=y+x
\end{align*}
and 
\begin{align*}
x+y=(a//b)(c//d)&=ac//bd\\
&=ca//db=(c//d)(a//b)=yx.
\end{align*}
Associativity of addition has already been shown, for multiplication we have
\begin{align*}
(xy)z=((a//b)(c//d))(e//f)&=(ac//db)(e//f) \\
&=ace//dbf \\
&=(a//b)(ce//df) 
=(a//b)((c//d)(e//f))=x(yz).
\end{align*}
For the additive and multiplicative identities, 
\[
x+0=(a//b)+(0//1)=(a1+b0)//b1=a//b=x
\]
and 
\[
x1=(a//b)(1//1)=a1//b1=a//b=x
\]
and since addition and multiplication is commutative, we also have $0+x=x$ and $1x=x$. Next, for the additive inverse, 
\[
x+(-x)=(a//b)+(-a//b)=(ab+(-ab))//bb=0//b^2=0
\]
and again, by commutativity, $(-x)+x=0$. For the distributive property, we have
\begin{align*}
x(y+z)&=(a//b)((c//d)+(e//f)) \\
&=(a//b)((cf+de)//df) \\
&=a(cf+de)//bdf \\
&=(acf+ade)//bdf \\
&=((acf+ade)//bdf)(b//b) \\
&=(acbf+bdae)//bdbf \\
&=(ac//bd)+(ae//bf) \\
&=(a//b)(c//d)+(a//b)(e//f)=xy+xz.
\end{align*}
It then follows by commutativity of multiplication that
\[
(y+z)x=x(y+z)=xy+xz=yx+zx
\]
Lastly for the multiplicative inverse, if $x\neq 0$ we have
\[
xx^{-1}=(a//b)(b//a)=ab//ab=1//1=1
\]
and by commutativity, $x^{-1}x=1$. \qed \\

\subsubsection{} Prove Lemma 4.2.7. (Note that, as in Proposition 2.2.13, you have to prove two different things: firstly, that \textit{at least} one of (a), (b), (c) is true; and secondly, that \textit{at most} one of (a), (b), (c) is true.) \\
\soln Let $x=a//b$ where $a,b$ are integers and $b$ is non-zero. Note that $(-n)//m=n//(-m)$ so we can suppose without loss of generality that $b$ is a positive integer. First, we show that at least one of (a), (b), (c) hold. By Lemma 4.1.5, $a$ is either zero, positive, or negative. Respectively, in each case of Lemma 4.1.5, it follows that (a), (b), (c) also hold---because if $a$ is zero, then $a//b=0//b=0$; if $a$ is positive, then $a//b$ is positive and; if $a$ is negative, then $a//b$ is negative. Next, we show that at most one of (a), (b), (c) is true. If (a) is true, then $a//b=0/1$ which implies that $a1=b0$ or $a=0$ which means that (b)
and (c) cannot be true because $a$ is not positive nor negative. If (b) is true, then $a//b$ is positive which implies that $a$ must be a positive integer because we have assumed $b$ to be positive. Thus, $a$ cannot be negative nor be equal to zero or that (b) and (c) is false. Similarly, If (c) is true, then $a//b$ is negative which implies that $a$ is negative and thus, cannot be positive nor be equal to zero. \qed\\

\subsubsection{} Prove Proposition 4.2.9. \\
\soln Let $x,y,z$ be rational numbers.
\begin{enumerate}
\item[(a)] This follows immediately from the trichotomy of the rationals (Lemma 4.2.7). Note that $x-y$ is an integer. If $x-y=0$, then $x=y$. If $x-y$ is positive, then $x>y$. Lastly, if $x-y$ is negative, then $x<y$. Lemma 4.2.7 guarantees that only one of these cases can be true for any given $x$ and $y$. 
\item[(b)] Suppose $x<y$, then $x-y$ is a negative rational. It follows that $-(x-y)=y-x$ is a positive rational number so $y>x$.
\item[(c)] Suppose $x<y$ and $y<z$, then $x-y=-a$ and $y-z=-b$ for some positive rationals $a$ and $b$. Adding the two equation yields $x-z=(x-y)+(y-z)=(-a)+(-b)=-(a+b)$, where $a+b$ is a positive rational. Hence, $x<z$.
\item[(d)] Suppose $x<y$, then $x-y=-a$ for some positive rational $a$. Note that $z-z=0$ so $(x-y)+(z-z)=(x+z)-(y+z)=-a+0=-a$ which implies that $x+z<y+z$. 
\item[(e)] Suppose $x<y$ and that $z$ is positive. We have $x-y=-a$ for some positive rational $a$. Multiplying both sides by $z$, we get $(x-y)z=xz-yz=-az$. Note that since $z$ is positive, $az$ is also positive and thus, $xz<yz$. 
\end{enumerate}
\qed \\ 

\subsubsection{} Show that if $x,y,z$ are rational numbers such that $x<y$ and $z$ is negative, then $xz>yz$. \\
\soln Suppose $x<y$ and $z$ is negative, then $x-y=-a$ for some positive rational $a$. We can also write $z=-b$ for some positive rational $b$. Multiplying the two equations together yields $(x-y)z=xz-yz=(-a)(-b)=ab$. Note that $ab$ is positive because both $a$ and $b$ are positive rationals. Hence, $xz>yz$ as desired. \qed \\

\subsection{Absolute value and exponentiation} 

\subsubsection{} Prove Proposition 4.3.3. (Hint: while all of these claims can be proven by dividing into cases, such as when $x$ is positive, negative, or zero, several parts of the proposition can be proven without such a tedious division into cases. For instance one can use earlier parts of the proposition to prove later ones.) \\
\soln 
\begin{enumerate}
\item[(a)] We split this into cases (according to the trichotomy of the rationals). If $x=0$, then $|x|=0\geq 0$. If $x$ is positive, then $x=x-0$ is positive which means $x>0$ so $|x|=x>0$. If $x$ is negative, then $-x=-x-0$ is positive which means $-x>0$ and thus $|x|=-x>0$. In any case, $|x|\geq 0$ for any rational $x$. 
Now, suppose $x=0$, then we have $|x|=0$ by definition. Conversely, suppose $|x|=0$. By what we have shown, if $x$ is positive or negative, then $|x|>0$ so we must have $x=0$. 
\item[(b)] Using 
\item[(c)] $(\implies)$ Suppose $-y\leq x\leq y$. We split this into three cases by the trichotomy of the rationals. If $x=0$ then $y\geq 0$ and $|x|=0$ so $y\geq |x|$. If $x$ is positive then $y\geq x$ and $|x|=x$ so $y\geq |x|$. If $x$ is negative then $x\geq -y$ which follows that $y\geq -x$. We also have $|x|=-x$ so $y\geq |x|$. In any case, we have $y\geq |x|$. $(\impliedby)$ Suppose $y\geq |x|$. Again, we split this into cases by the trichotomy of the rationals. If $x=0$ or if $x$ is positive then $|x|=x$ so $y\geq x$. If $x$ is negative then $|x|=-x$ so $y\geq -x$ and equivalently, $-y\leq x$. Hence, we have $-y\leq x\leq y$ as desired.
\item[(d)] We split this into cases. First we consider the case if at least one of $x$ and $y$ is zero, then if both are positive or negative, and if $x$ is positive while $y$ is negative. (1) If at least one of $x$ or $y$ is zero, then $xy=0$ which follows that $|xy|=0=|x||y|$. (2) If both $x$ and $y$ are positive then $xy$ is also positive and so $|xy|=xy=|x||y|$. (3) If both $x$ and $y$ are negative then $xy$ is positive and also $|x|=-x$ and $|y|=-y$. We then have $|xy|=xy=(-x)(-y)=|x||y|$. (4) If $x$ is positive while $y$ is negative, then $xy$ is negative and we have $|x|=x$ and $|y|=-y$. It follows that $|xy|=-xy=x(-y)=|x||y|$. In any case we have $|xy|=|x||y|$ for any rationals $x$ and $y$.
\item[(e)] This follows directly from (a). We have $d(x,y)=|x-y|\geq 0$. Moreover, $d(x,y)=|x-y|=0$ iff $x-y=0$ or that $x=y$.
\item[(f)] This follows from (d). Note that $|-1|=1$. Observe that \begin{align*}
d(x,y)=|x-y|=|x-y|1&=|x-y||-1|\\
&=|-(x-y)|=|y-x|=d(y,x).
\end{align*}
\item[(g)] This follows from (b). Using the triangle inequality, we have
\begin{align*}
\qquad d(x,z)=|x-z|=|(x-y)+(y-z)|\leq |x-y|+|y-z|=d(x,y)+d(y,z). 
\end{align*}
\end{enumerate} \qed\\

\subsubsection{} Prove the remaining claims in Proposition 4.3.7. \\
\soln 
\begin{enumerate}
\item[(a)] Suppose $x=y$, then $x-y=0$ so $d(x,y)=|x-y|=0\leq\varepsilon$ for all rationals $\varepsilon>0$. Conversely, suppose $x$ is $\varepsilon$-close to $y$ for every $\varepsilon>0$. If we suppose on the contrary that $x\neq y$, then $x=y+a$ for some rational $a$. It follows that $d(x,y)=|x-y|=|a|>|a|/2$, where $|a|/2$ is a positive rational, a contradiction to our assumption.
\item[(b)] Suppose $\varepsilon>0$ and $x$ is $\varepsilon$-close to $y$, then $d(x,y)=|x-y|\leq\varepsilon$. Recall that distance is symmetric so $d(y,x)=d(x,y)\leq\varepsilon$, that is, $y$ is also $\varepsilon$-close to $x$.
\item[(c)] Let $\varepsilon,\delta>0$. Suppose $x$ is $\varepsilon$-close to $y$ and $y$ is $\delta$-close to $z$. We have $d(x,y)\leq \varepsilon$ and $d(y,z)\leq \delta$. Using the triangle inequality for distance, we have $d(x,z)\leq d(x,y)+d(y,z)=\varepsilon+\delta$. Thus, $x$ and $z$ are $(\varepsilon+\delta)$-close.
\end{enumerate}




\newpage

\section{The real numbers}

\subsection{Cauchy sequences}

\subsubsection{} Prove Lemma 5.1.15. (Hint: use the fact that $a_n$ is eventually $1$-steady, and thus can be split into a finite sequence and a $1$-steady sequence. Then use Lemma 5.1.14 for the finite part. Note there is nothing special about the number $1$ used here; any positive number would have sufficed.) \\
\soln Suppose $(a_n)_{n=1}^\infty$ is a Cauchy sequence. Then $(a_n)_{n=1}^\infty$ is eventually $\varepsilon$-steady for every $\varepsilon>0$, i.e., there exists a natural number $N$ such that $(a_n)_{n=N}^\infty$ is $\varepsilon$-steady. In particular, it is $1$-steady so we have $d(a_n,a_N)=|a_n-a_N|\leq 1$ for all $n\geq N$. By Proposition 4.3.3, we have $-1\leq a_n-a_N\leq 1\implies -1+a_N\leq a_n\leq 1+a_N$. Recall that $-|a_N|\leq a_N\leq |a_N|$ so we have $-1-|a_N|\leq -1+a_N$ and $1+a_N\leq 1+|a_N|$. Thus we have $-(1+|a_N|)\leq a_n\leq 1+|a_N|$ and equivalently, $|a_n|\leq 1+|a_N|$, i.e., $(a_n)_{n=N}^\infty$ is bounded by $1+|a_N|$. 
\[
\overbrace{a_1,a_2,\dots,a_{N-1}}^{\text{bounded by } M}, \overbrace{a_N,a_{N+1},a_{N+3},a_{N+4}\dots}^{\text{1-steady, bounded by } 1+|a_N|}
\]

On the other hand, we also have to worry about the finite sequence we have left behind that is $a_1,a_2,\dots, a_{N-1}$ but by Lemma 5.1.14, we know that this finite sequence is bounded by some rational $M\geq 0$. In other words, $|a_i|\leq M$ for all $1\leq i\leq N-1$. Then, we can simply choose which of the two, $1+|a_N|$ or $M$, is larger then we will have guaranteed that the whole sequence is bounded. Hence, $|a_n|\leq\max(1+|a_N|,M)$ for all $n\geq 1$ and we are done. \qed \\

\subsection{Equivalent Cauchy sequences} 

\subsubsection{} Show that if $(a_n)_{n=1}^\infty$ and $(b_n)_{n=1}^\infty$ are equivalent sequences of rationals, then $(a_n)_{n=1}^\infty$ is a Cauchy sequence if and only if $(b_n)_{n=1}^\infty$ is a Cauchy sequence. \\
\soln Let $(a_n)_{n=1}^\infty$ and $(b_n)_{n=1}^\infty$ be equivalent sequences of rationals. $(\implies)$ Let $\varepsilon>0$. Suppose $(a_n)_{n=1}^\infty$ is a Cauchy sequence, then it is eventually $\varepsilon/3$-steady. That is, there exists $N_1\geq 0$ such that $a_j$ and $a_k$ are $\varepsilon/3$-close for all $j,k\geq N_1$. Furthermore, since $(a_n)_{n=1}^\infty$ and $(b_n)_{n=1}^\infty$ are equivalent, there exists $N_2\geq 0$ such that $a_n$ and $b_n$ are $\varepsilon/3$-close. If we take $N=\max (N_1,N_2)$, then we can guarantee that $a_j$ and $a_k$ are always $\varepsilon/3$-close and $a_n$ and $b_n$ are always $\varepsilon/3$-close for any $j,k,n\geq N$ because $N=\max (N_1,N_2)\geq N_1$ and $N=\max (N_1,N_2)\geq N_2$. In summary, note that 
\begin{align*}
b_j \text{ is $\varepsilon/3$-close to } a_j \tag1 \\
a_j \text{ is $\varepsilon/3$-close to } a_k \tag2 \\
a_k \text{ is $\varepsilon/3$-close to } b_k \tag3
\end{align*}
for any $j,k\geq N$. By Proposition 4.3.7, $b_j$ is $2\varepsilon/3$-close to $a_k$ by (1) and (2), and then by (3) $b_j$ is $\varepsilon$-close to $b_k$. This means that $(b_n)_{n=1}^\infty$ is eventually $\varepsilon$-steady and hence, a Cauchy sequence. $(\impliedby)$ The converse direction is proved using the same argument, just with the $a$'s and $b$'s switched. \qed \\ 

\subsubsection{} Let $\varepsilon>0$. Show that if $(a_n)_{n=1}^\infty$ and $(b_n)_{n=1}^\infty$ are eventually $\varepsilon$-close, then $(a_n)_{n=1}^\infty$ is bounded if and only if $(b_n)_{n=1}^\infty$ is bounded. \\
\soln Let $\varepsilon>0$ and suppose that $(a_n)_{n=1}^\infty$ and $(b_n)_{n=1}^\infty$ are eventually $\varepsilon$-close. In particular, $(a_n)_{n=1}^\infty$ and $(b_n)_{n=1}^\infty$ are eventually $1$-close which means there exists an $N\geq 0$ such that $|a_n-b_n|\leq 1$ for all $n\geq N$. Suppose further that $a_n$ is bounded, i.e., for every $n\geq 1$, $|a_n|\leq M$ for some positive rational $M$. Now, observe that if $n\geq N$, we have
\[
|b_n|=|(a_n)-(a_n-b_n)|\leq |a_n|+|a_n-b_n|\leq M+1,
\]
so $(b_n)_{n=N}^\infty$ is bounded. Note that we have only shown boundedness for the tail of the sequence. We also have to show that the finite sequence $b_1,b_2,\dots, b_{N-1}$ is bounded but this is easy because Lemma 5.1.14 asserts that this finite sequence must be bounded by some rational $M'$. Then we can simply take the larger of the two, $M+1$ or $M'$, so that $|b_n|\leq\max (M+1, M')$ for all $n\geq 1$ and we are done. \qed \\

\subsection{The construction of the real numbers} 

\subsubsection{} Prove Proposition 5.3.3. (Hint: you may find Proposition 4.3.7 to be useful.) \\
\soln Let $x=\LIM a_n$, $y=\LIM b_n$, and $z=\LIM c_n$ be real numbers and $\varepsilon>0$. (\textbf{Reflexivity}) Since $x$ is a real number, the sequence $a_n$ is a Cauchy sequence. Note that $(a_n)_{n=1}^\infty$ is equivalent to itself because $|a_n-a_n|=0\leq\varepsilon$ for any $\varepsilon>0$. Thus, $x=\LIM a_n=x$. (\textbf{Symmetry}) Suppose $x=y$, then $(a_n)_{n=1}^\infty$ and $(b_n)_{n=1}^\infty$ are equivalent Cauchy sequences, i.e., there exists an $N\geq 0$ such that $a_n$ is $\varepsilon$-close to $b_n$ for all $n\geq N$. By Proposition 4.3.7, it follows that $b_n$ is also $\varepsilon$-close to $a_n$. Thus, $y=\LIM b_n=\LIM a_n=x$. (\textbf{Transitivity}) Suppose $x=y$ and $y=z$. Then $(a_n)_{n=1}^\infty$ and $(b_n)_{n=1}^\infty$ are equivalent Cauchy sequences which means that there exists an $N_1\geq0$ such that $a_n$ and $b_n$ are $\varepsilon/2$-close for all $n\geq N_1$. Similarly, since $(b_n)_{n=1}^\infty$ and $(c_n)_{n=1}^\infty$ are equivalent Cauchy sequences, there exists an $N_2\geq0$ such that $b_n$ and $c_n$ are $\varepsilon/2$-close for all $n\geq N_2$. By taking $N=\max (N_1,N_2)$, it follows by Proposition 4.3.7 that $a_n$ and $c_n$ are $\varepsilon$-close. Thus, $(a_n)_{n=1}^\infty$ and $(c_n)_{n=1}^\infty$ are equivalent Cauchy sequences and so, $x=\LIM a_n=\LIM c_n=z$. \qed \\

\subsubsection{} Prove Proposition 5.3.10 (Hint: again, Proposition 4.3.7 may be useful.) \\
\soln Let $x=\LIM a_n$, $y=\LIM b_n$, and $x'=\LIM a'_n$ be real numbers. \\

For the \textbf{first claim}, we need to show that given Cauchy sequences $(a_n)_{n=1}^\infty$ and $(b_n)_{n=1}^\infty$, their product $(a_nb_n)_{n=1}^\infty$ is also Cauchy. We need to show that for every $\varepsilon>0$, the sequence $(a_nb_n)_{n=1}^\infty$ is eventually $\varepsilon$-steady. We know that $(a_n)_{n=1}^\infty$ is eventually $\varepsilon_a$-steady and $(b_n)_{n=1}^\infty$ is eventually $\varepsilon_b$-steady because they are Cauchy. This means that there exists natural numbers $N_1,N_2\geq 0$ such that 
\begin{center}
    $a_j$ is $\varepsilon_a$-close to $a_k$, \\
    $b_j$ is $\varepsilon_b$-close to $b_k$
\end{center}
for all $j,k\geq N_1,N_2$. By Proposition 4.3.7h, we know that 
\begin{center}
$a_jb_j$ is $(\varepsilon_a|b_j|+\varepsilon_b|a_j|+\varepsilon_a\varepsilon_b)$-close to $a_kb_k$. 
\end{center}
The natural thing to do next is to find $\varepsilon_a$ and $\varepsilon_b$ such that $\varepsilon_a|b_j|+\varepsilon_b|a_j|+\varepsilon_a\varepsilon_b\leq \varepsilon$ however as one would discover doing some calculations, we might run into a term which has $|a_j|$ or $|b_j|$ in the denominator. This is not good because we have not guaranteed that every $a_n$ and $b_n$ is non-zero. To remedy this, we use the fact that Cauchy sequences are bounded. By Lemma 5.1.15, $(a_n)_{n=1}^\infty$ and $(a_n)_{n=1}^\infty$ are bounded, i.e., there exists rationals $M_1,M_2>0$ such that $a_n\leq M_1$ and $b_n\leq M_2$ for all $n\geq 1$. \\

Going back, we want $\varepsilon_a|b_j|+\varepsilon_b|a_j|+\varepsilon_a\varepsilon_b\leq \varepsilon$. Since there are three terms, we can put a constraint that each term is less than or equal to $\varepsilon/3$ so that the total equals $\varepsilon$. We have
\begin{align*}
    \varepsilon_a|b_j|\leq \varepsilon_aM_2 &\leq \varepsilon/3, \tag1\\
    \varepsilon_b|a_j|\leq \varepsilon_bM_1&\leq \varepsilon/3, \text{ and} \tag2\\
    \varepsilon_a\varepsilon_b&\leq \varepsilon/3 \tag3
\end{align*}
To simplify things, we can take $\varepsilon_a=\varepsilon/3M_2$ so that (1) still holds. Then we can substitute $\varepsilon_a$ to (3) which gives us the following constraint for $\varepsilon_b$,
\[
\varepsilon_b\leq \varepsilon/3M_1 \text{ and } \varepsilon_b\leq \varepsilon/3\varepsilon_a=M_2.
\]
We can then choose $\varepsilon_b=\min (\varepsilon/3M_1, M_2)$ so that (2) and (3) still hold. Note that $M_1$ and $M_2$ are positive rationals so there are no divisions by zero. Furthermore, since $\varepsilon, M_1, M_2$ are all positive rationals, it follows that $\varepsilon_a$ and $\varepsilon_b$ are positive rationals as well. \\

\underline{We now finish the proof}. Let $\varepsilon>0$. Since $(a_n)_{n=1}^\infty$ and $(b_n)_{n=1}^\infty$ are Cauchy, there exists natural number $N_1,N_2\geq 0$ such that 
\begin{center}
    $a_j$ is $\varepsilon_a$-close to $a_k$, \\
    $b_j$ is $\varepsilon_b$-close to $b_k$
\end{center}
for all $j,k\geq N_1,N_2$ where $\varepsilon_a,\varepsilon_b$ are any positive rationals. Take $N=\max (N_1,N_2)$, then by Proposition 4.3.7h, we know that 
\begin{center}
$a_jb_j$ is $(\varepsilon_a|b_j|+\varepsilon_b|a_j|+\varepsilon_a\varepsilon_b)$-close to $a_kb_k$. 
\end{center}
whenever $j,k\geq N$. In particular, choose $\varepsilon_a=\varepsilon/3M_2$ and $\varepsilon_b=\min (\varepsilon/3M_1, M_2$) where $M_1$ and $M_2$ are bounds of $(a_n)_{n=1}^\infty$ and $(b_n)_{n=1}^\infty$, respectively. As we have previously justified, our choice of $\varepsilon_a$ and $\varepsilon_b$ guarantees that (1), (2), and (3) hold so 
\[
\varepsilon_a|b_j|+\varepsilon_b|a_j|+\varepsilon_a\varepsilon_b
\leq \frac{\varepsilon}{3}+\frac{\varepsilon}{3}+\frac{\varepsilon}{3}=\varepsilon
\]
It then follows by Proposition 4.3.7e that $a_jb_j$ is $\varepsilon$-close to $a_kb_k$ for all $j,k\geq N$. This means that $(a_nb_n)_{n=1}^\infty$ is eventually $\varepsilon$-steady and thus, Cauchy. Hence, $xy=\LIM a_nb_n$ is a real number. \qed \\

Next we prove the \textbf{second claim}. Suppose $x=x'$, then $(a_n)_{n=1}^\infty$ and $(a'_n)_{n=1}^\infty$ are equivalent Cauchy sequences. This means that for all $\varepsilon'>0$, the sequences are eventually $\varepsilon'$-close, i.e., for all $\varepsilon'>0$ there exists $N\geq 0$ such that $a_n$ and $a'_n$ are $\varepsilon'$-close for all $n\geq N$. Let $\varepsilon>0$. We want to show that $(a_nb_n)_{n=1}^\infty$ and $(a'_nb_n)_{n=1}^\infty$ are eventually $\varepsilon$-close. Note that whenever $b_n=0$, $a_nb_n=0=a'_nb_n$ so they are trivially $\varepsilon$-close. Suppose $b_n$ is non-zero, then by Proposition 4.3.7g, $a_nb_n$ is $\varepsilon'|b_n|$-close to $a'_nb_n$ whenever $n\geq N$. Since $\varepsilon'$ is an arbitrary positive rational, we can choose $\varepsilon'=\varepsilon/|b_n|$ so that $\varepsilon'|b_n|=\varepsilon$. It follows that $a_nb_n$ is $\varepsilon$-close to $a'_nb_n$ whenever $n\geq N$. This means that $(a_nb_n)_{n=1}^\infty$ and $(a'_nb_n)_{n=1}^\infty$ are eventually $\varepsilon$-close sequences and thus, equivalent Cauchy sequences (recall that we have already shown that products of Cauchy sequences are cauchy). Hence, $xy=x'y$ and we are done. \qed \\

\textit{Remark.} Since multiplication is commutative, it suffices to prove the substitution property for one side because supposing $y=y'$, then $xy=yx=y'x=xy'$. \\

\subsubsection{} Let $a,b$ be rational numbers. Show that $a=b$ if and only if $\LIM a=\LIM b$ (i.e., the Cauchy sequences $a,a,a,a,\dots$ and $b,b,b,b,\dots$ are equivalent if and only if $a=b$). This allows us to embed the rational numbers inside the real numbers in a well-defined manner. \\
\soln $(\implies)$ Suppose $a=b$, then $a-b=0$ which means that $|a-b|=0\leq \varepsilon$ for any $\varepsilon>0$. This means that $(a)_{n=1}^\infty$ and $(b)_{n=1}^\infty$ are equivalent Cauchy sequences, and thus $\LIM a=\LIM b$. $(\impliedby)$ Suppose $\LIM a=\LIM b$, then for all $\varepsilon>0$, there exists $N\geq0$ such that $a$ and $b$ are $\varepsilon$-close for all $n\geq N$. However since $a$ and $b$ do not depend on the choice of $N$ (because they are constant), they are simply $\varepsilon$-close for all $\varepsilon>0$. Then by Proposition 4.3.7, we must have $a=b$ as desired. \qed \\


\subsubsection{} Let $(a_n)_{n=0}^\infty$ be a sequence of rational numbers which is bounded. Let $(b_n)_{n=0}^\infty$ be another sequence of rational numbers which is equivalent to $(a_n)_{n=0}^\infty$. Show that $(b_n)_{n=0}^\infty$ is also bounded. (Hint: use Exercise 5.2.2.) \\
\soln Suppose $(a_n)_{n=0}^\infty$ and $(b_n)_{n=0}^\infty$ are equivalent Cauchy sequences which by Definition 5.2.6 means that they are eventually $\varepsilon$-close. Then by Exercise 5.2.2, since $(a_n)_{n=0}^\infty$ is bounded, it follows that $(b_n)_{n=0}^\infty$ is bounded as well. \qed \\


\subsubsection{} Show that $\LIM 1/n=0$. \\
\soln Note that $0$ is a rational number so by definition, $0=\LIM 0$. It then suffices to show that the sequences $(1/n)_{n=1}^\infty$ and $(0)_{n=0}^\infty$ are equivalent. Let $\varepsilon>0$. We need to show that there exist an $N\geq0$ such that for all $n\geq N$, we have
\[
\left|\frac{1}{n}-0\right|=\left|\frac{1}{n}\right|\leq\varepsilon.
\]
Note that $1/n$ is positive for all $n\geq 1$ so $|1/n|=1/n$. Observe that $n\geq N$ is equivalent to $1/n\leq 1/N$. Thus we can choose $N$ such that $1/N\leq \varepsilon$ or equivalently $N\geq 1/\varepsilon$. Recall that Proposition 4.4.1 guarantees that there exists such a natural number $N$ (and it is unique). Thus, we have
\[
\left|\frac{1}{n}-0\right|=\left|\frac{1}{n}\right|=\frac{1}{n}\leq\frac{1}{N}\leq\varepsilon.
\]
Hence, $(1/n)_{n=1}^\infty$ and $(0)_{n=0}^\infty$ are equivalent Cauchy sequences and so, $\LIM 1/n=0$. \qed \\


\subsection{Ordering the reals}

\subsubsection{} Prove Proposition 5.4.4. (Hint: if $x$ is not zero, and $x$ is the formal limit of some sequence $(a_n)_{n=1}^\infty$, then this sequence cannot be eventually $\varepsilon$-close to the zero sequence $(0)_{n=1}^\infty$ for every single $\varepsilon>0$. Use this to show that the sequence $(a_n)_{n=1}^\infty$ is eventually either positively bounded away from zero or negatively bounded away from zero.) \\
\soln First we show that at least one of the statements hold. Let $x=\LIM a_n$ be a real number, then $(a_n)_{n=1}^\infty$ is a Cauchy sequence. Suppose $x\neq0$, then $(a_n)_{n=1}^\infty$ cannot be equivalent to the zero sequence $(0)_{n=1}^\infty$. This means that there exists a rational $\varepsilon>0$ such that for any natural number $N$, there exists an $n\geq N$ such that $|a_n-0|=|a_n|>\varepsilon$. \\
We then claim that every Cauchy sequence not equivalent to the zero sequence is either positively bounded away from zero or negatively bounded away from zero. \textit{not finished}


\subsubsection{} Prove the remaining claims in Proposition 5.4.7. \\
\soln Let $x,y,z$ be real numbers. 
\begin{enumerate}
\item[(a)] (Order trichotomy) This follows from the trichotomy of the reals. Note that $x-y$ is a real number so exactly one of the following is true: (1) $x-y=0$, (2) $x-y$ is positive, or (3) $x-y$ is negative. If (1) holds, then $x=y$. If (2) holds, then $x>y$. And if (3) holds, then $x<y$. 
\item[(b)] (Order is anti-symmetric) Suppose $x<y$, then $x-y$ is a negative real number. It follows that $-(x-y)=y-x$ is positive so $y>x$ as desired. The converse direction is proven similarly.
\item[(c)] (Order is transitive) Suppose $x<y$ and $y<z$. Then $y-x$ and $z-y$ are positive real numbers. Observe that $(y-x)+(z-y)=z-x$ is also a positive real number by Proposition 5.4.4 so we have $x<z$.
\item[(d)] (Addition preserves order) Suppose $x<y$, then $y-x$ is positive. Note that $z-z=0$ so $(y+z)-(x+z)=(y-x)+(z-z)=y-x$ is also positive so $x+z<y+z$.
\item[(e)] (Positive multiplication preserves order) \textit{already been shown in the text} \\
\end{enumerate}


\subsubsection{} Show that for every real number $x$ there is exactly one integer $N$ such that $N\leq x<N+1$. *(This integer $N$ is called the \textit{integer part} of $x$, and is sometimes denoted $N=\lfloor x\rfloor$.) \\
\soln 



\end{document}